\subsection{Beat tracking experiment} % \label{sec:beat}
So far, to the best of our knowledge, no beat tracking has been attempted on Indian and Turkish music. Hence, we chose two acknowledged approaches presented in the last decade \cite{klapuri:06:meter,ellis:07:beat} for evaluation. 

\citeA{ellis:07:beat}: \acrshort{ELL} algorithm.

BT evaluation: For our evaluations, we can compare the pulsations estimated by the algorithms with our ground truth annotations at all three metrical levels to determine if the large possible tempo ranges cause the beat to be tracked at different levels of the meter. Using the beat ground truth annotations from both the Turkish and Carnatic low-level annotated datasets, we can define the median inter-annotation time-span as $T_a$ for each music piece. From the estimates obtained from \acrshort{KLA} for downbeats, beats and subdivision pulses on a specific piece, we define the following time-spans: let $T_m$ denote the median measure time-span, $T_b$ the median beat time-span, and $T_s$ the median subdivision time-span. Let $T_{b,\acrshort{ELL}}$ denote the median beat time-span as estimated by the \acrshort{ELL} algorithm. The median is computed to obtain an estimate that is robust to spurious deviations. 
%%%%%%%%%%%%%%%%%%%%%%%%%%%%%%%%%%%%%%%%%%%%%%%%%%%%%%%%%%%%%%%%%%%%%%%%%%%%%%%%%%%%%%%%%%%%%%%%%%%%%%%%%%%%%%%%%%%%%%%%%%%%%%%
%\begin{figure*}[htb!]
  %\centering
 %\subfloat[\acrshort{KLA}]{\label{tempHist_C_am} \includegraphics[width=0.255\columnwidth]{./fig/TempHist11.png}}
 %\subfloat[\acrshort{KLA}]{\label{tempHist_C_ab} \includegraphics[width=0.245\columnwidth]{./fig/TempHist12.png}}
 %\subfloat[\acrshort{KLA}]{\label{tempHist_C_as} \includegraphics[width=0.245\columnwidth]{./fig/TempHist13.png}}
 %\subfloat[\acrshort{ELL}]{\label{tempHist_C_ell} \includegraphics[width=0.245\columnwidth]{./fig/TempHist14.png}}\\
%\caption{Relative time-span histograms on Carnatic low-level-annotated dataset. The panels (a), (b), (c) correspond to \acrshort{KLA} algorithm. Panel (d) corresponds to the \acrshort{ELL} algorithm. The ordinate is the fraction of pieces (in percentage) and is normalized to sum to 100\%.}
%\label{fig:tempHistC}
%\end{figure*}
%\begin{figure*}[htb!]
  %\centering
 %\subfloat[\acrshort{KLA}]{\label{tempHist_T_am} \includegraphics[width=0.255\columnwidth]{./fig/TempHist21.png}}
 %\subfloat[\acrshort{KLA}]{\label{tempHist_T_ab} \includegraphics[width=0.245\columnwidth]{./fig/TempHist22.png}}
 %\subfloat[\acrshort{KLA}]{\label{tempHist_T_as} \includegraphics[width=0.245\columnwidth]{./fig/TempHist23.png}}
 %\subfloat[\acrshort{ELL}]{\label{tempHist_T_ell} \includegraphics[width=0.245\columnwidth]{./fig/TempHist24.png}}\\
%\caption{Relative time-span histograms on Turkish low-level-annotated dataset. The panels (a), (b), (c) correspond to \acrshort{KLA} algorithm. Panel (d) corresponds to the \acrshort{ELL} algorithm. The ordinate is the fraction of pieces (in percentage) and is normalized to sum to 100\%.}
%\label{fig:tempHistT}
%\end{figure*}

In Hindustani music, beats of a piece are not well defined, since it depends also on the \textit{lay}. The beats could be defined at the \gls{matra} level for slow \textit{lay} and at \gls{vibhaag} level for faster pieces and hence there exists no unambiguous definition of beats \cite[p.~91]{clayton:00:time}. Due to this ambiguity which needs a deeper exploration, we do not explore beat tracking for Hindustani music formally in this paper.

We evaluate beat tracking on the Turkish low-level-annotated dataset and the Carnatic low-level-annotated dataset introduced in Section~\ref{sec:collections:lowlevel}, using \acrshort{KLA} and \acrshort{ELL} beat tracking algorithms. Since \acrshort{KLA} algorithm estimates meter at three levels, we evaluate beat tracking performance of the algorithm with the annotated pieces at the subdivision, beat and measure level to ascertain whether the algorithm tracks the annotated beats on any of those levels. We evaluate in how far the estimated beats are accurate in terms of the estimated time-spans, and if they are correctly aligned to the phase of the annotations. We further discuss various kinds of errors observed in the beat tracking results. 

We use the different time-span definitions from Section~\ref{sec:beat:evalCriteria} to obtain histograms of estimated relative tempo for both low-level-annotated datasets, as shown in Figures~\ref{fig:tempHistC} and~\ref{fig:tempHistT}. The figure shows the histogram of the ratios $T_a/T_m$, $T_a/T_b$, $T_a/T_s$, obtained from the estimates by the \acrshort{KLA} algorithm, and $T_a/T_{b,\acrshort{ELL}}$ for the estimation by \acrshort{ELL} algorithm. A ratio of one would imply that the annotation time-span and the estimated pulse time-span matched exactly, and the pulsation is thus tracked at the annotated level. A ratio of, \textit{e.g} 2, would indicate that the estimated tempo is two times faster than the annotated one. 

From the panels (a)-(c) of Figures~\ref{fig:tempHistC} and~\ref{fig:tempHistT}, we see that in a majority of pieces in both the low-level-annotated datasets, the \acrshort{KLA} algorithm tracks the annotated beats at the beat level, as indicated by Figures (\ref{tempHist_C_ab}) and (\ref{tempHist_T_ab}). It is also clear from the figure that the measure time-spans estimated by the algorithm are at least twice as long as the annotated beat time-spans and hence we conclude that the estimated downbeats never correspond to the annotated beats in both datasets. Further, we see that the estimated subdivision time-spans are much shorter than the annotated beat time-spans. Based on these observations, we present the beat tracking results using only the estimated beats for \acrshort{KLA} algorithm. In general, we can also see that there is a tendency towards tempo under-estimation for Turkish music (Figure~(\ref{tempHist_T_ab})), and towards tempo over-estimation for Carnatic music (Figure~(\ref{tempHist_C_ab})). 
\begin{table}
\begin{centering}
\begin{tabular}{@{}lrr@{}}
\toprule
Metric & $\mathrm{\acrshort{KLA}}_{\mathrm{b}}$ & \acrshort{ELL}\tabularnewline \midrule
f-measure(\%) &  46.73 &  23.27\tabularnewline
Information Gain (bits) & 1.09 &  0.4254\tabularnewline
CMLt(\%) & 34.85 & 24.78\tabularnewline \bottomrule
\end{tabular}
\par\end{centering}
\caption{Beat Tracking Results for Carnatic music, $\mathrm{\acrshort{KLA}}_{\mathrm{b}}$ indicates that the results are reported using the beat pulsation estimates of \acrshort{KLA} algorithm}\label{tab:jnmreval:btKLA}
\end{table}
%%%

The beat tracking performance is shown in Table~\ref{tbl:BeatTrack}, averaged over the songs of the individual collections. In the table, $\mathrm{\acrshort{KLA}}_{\mathrm{b}}$ refers to evaluating using the beat estimates by the \acrshort{KLA} algorithm. \acrshort{ELL} refers to the performance using Ellis beat tracking algorithm. $\mathrm{\acrshort{KLA}}_{\mathrm{b}}$ provides us with the best accuracy while beat tracking on Turkish music seems to work slightly better than for Carnatic music. However, accuracies are low in general, with \textit{e.g.} CMLt accuracies of 44.01\% and 34.85\% using $\mathrm{\acrshort{KLA}}_{\mathrm{b}}$ on Turkish and Carnatic low-level-annotated datasets, respectively. This indicates that we are able to obtain an absolutely identical sequence in less than half of the cases on both datasets. Indeed, only 33\% of the samples in the Carnatic dataset, and 29\% of the samples in Turkish dataset have a CMLt higher than 50\%, which means that only a third of the pieces are absolutely correctly tracked at least in more than half of their duration. Information gain values are very low especially for Carnatic, which indicates a lot of sequences occur that are not simply related to the annotation (e.g. by tempo halving/doubling, or off-beat). In fact, only 21\% of Carnatic, and 34\% of Turkish low-level-annotated datasets have an Information Gain value larger than 1.5 bits. This value was established as a threshold for perceptually acceptable beat tracking~\cite{zapata:12:beat}. Even though this value was established on Eurogenetic popular music, the low values for Information Gain in Table~\ref{tbl:BeatTrack} imply that a large part of the beat tracking results is not perceptually acceptable. This seems to stand in conflict with the observation made in Figures~\ref{fig:tempHistC} and~\ref{fig:tempHistT} that in a majority of cases \acrshort{KLA} seems to get a median tempo that is related to the annotation. We observed however that many beat estimations suffer from unstable phase and tempo, which causes low values especially for Information Gain and CMLt. 

\acrshort{ELL} algorithm uses the tempo estimated by \acrshort{SRI} approach \cite{ajay:12:beatWkShop}, and uses a tempo weighting function which peaks at 90 bpm. Since the median inter-annotation time-span in Turkish low-level-annotated dataset corresponds to 147 bpm, the tempo estimate is not accurate and hence the CMLt performance is poor. The median inter-annotation time-span in Carnatic dataset corresponds to 85 bpm, which is closer to the peak of the weighting function than for Turkish music. We can also see this from Figures~(\ref{tempHist_C_ell}) and (\ref{tempHist_T_ell}), where we observe that \acrshort{ELL} tracks the tempo of the annotated beats in a majority of cases in Carnatic, while a majority of pieces in the Turkish dataset are tracked with slower tempo compared to the annotations. In many cases, the estimated beats tended to drift away from the annotations. This is initially caused by the resolution of the tempo estimate (about 11.6 ms, as used by \citeA{davies:07:beat}), which can result in small deviations from the true tempo. While such errors should be compensated for in the \acrshort{ELL} algorithm by the matching of the beat estimations to the accent signal, this compensation seems to fail in many cases especially for Carnatic music. 

In order to check if beat tracking tends to happen on the off-beat for some songs in the collections, we took advantage of the properties of the evaluation measures (see Section~\ref{sec:models:bt}). Since estimations with a constant phase misalignment are practically the only reason for the f-measure to take values close to zero, we located the appearance of those songs. Interestingly, for none of the Turkish music samples such a case appeared which shows that at least in the small available dataset there is no ambiguity regarding phase alignment. However, for Carnatic music, phase alignment caused problems for both \acrshort{ELL} and \acrshort{KLA}, revealed by a large number of samples with very low f-measure (<15\%). This effect is partly related to a non-zero \gls{edupu} (a phase shift of the start of the composition relative to the sama - see Section~\ref{sec:problemdef:bg}). In such a case, the composition starts not on the sama but 2, or 4, or 6 \glspl{akshara} after \gls{sama}. If the \gls{edupu} is 2 or 6 \glspl{akshara}, the accents of the composition in the piece are more likely to be on the off-beat and hence the beat tracker tracks the off-beat, with the tracked beat instant lying half way in between the two annotated beat instants. In general, we find poorer performance with pieces with a non-zero \gls{edupu}. In Carnatic dataset we have a total of 11 pieces with a non-zero \gls{edupu}, and the average CMLt accuracy values using $\mathrm{\acrshort{KLA}}_{\mathrm{b}}$ for those pieces was observed to be clearly lower than the average (CMLt: 20.19\% instead of 34.85\%). 
% (AH: Careful, this paragraph uses number of estimated beats, the previous inter beat time-spans. These two are highly correlated. If we do not use a graphical representation to show tempo relations, we should avoid this duality.)