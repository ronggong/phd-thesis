We summarize and discuss the key results of the evaluation presented in this section. The results provide us with useful insights to indicate promising directions for further work. At the outset, the results indicate that the performance of evaluated approaches is not adequate for the presented tasks, and that methods that are suitable to tackle the culture specific challenges in computational analysis of rhythm need to be developed. 

Cycle length estimation is challenging in Indian art music since cycles of different lengths exist at different time-scales. Although we defined the most important cycle to be at the \gls{avart}/\gls{avartana} level, the other cycles, mainly at the beat and subdivision level, also provide useful rhythm related information. The evaluated approaches \acrshort{PIK} and \acrshort{GUL} estimate the subdivision meter and time signature. This is possible to an acceptable level of accuracy, when restricting to a subset of rhythm classes with relatively simple subdivision meters. Though they do not provide a complete picture of the meter, they estimate the underlying metrical structures at short time scales and can be used as pre-processing steps for estimating longer and more complex cycles. 

Both \acrshort{SRI} and \acrshort{KLA} aimed to estimate the longer cycle lengths but show a performance that is inadequate for any practical application involving cycle length estimation. Tempo estimation and beat tracking have a significant effect on cycle length estimation, especially in the self-contained approaches and also need to be explored further. The comparative approaches show that the applied signal features capture important aspects of rhythm but are not sufficient to be used standalone for cycle estimation. A combination of self-contained and comparative approaches might provide useful insights into rhythm description of Indian art music through mutual reinforcement. %Developing methods that take the culture-specific properties of rhythm into account is therefore necessary to proceed towards a reliable computational rhythm analysis.

Downbeat tracking was explored using \acrshort{HOC} and \acrshort{DAV} algorithms. The downbeat detectors evaluated here needed an estimation of beats and the cycle length of the piece, which in themselves are difficult to estimate. Since downbeat information can help in estimating the cycle length and also beat tracking, a joint estimation of the beat, cycle length, and downbeat might be a potential solution since each of these parameters are mutually useful for estimating the others. A combination of bottom up and top down knowledge based approach which performs a joint estimation of these parameters is to be explored further, using models that better represent the underlying metrical structures. 

Long \gls{avart} cycle is a significant challenge in Hindustani music. For \glspl{taal} with a very long cycle duration, estimating the correct metrical level is essential and methods that aim at tracking short time-span pulsation will be not adequate due to the grouping structure of the \gls{taal}. With a wide variety of rhythms, coupled with the perceptual \gls{edupu}, Carnatic music poses a difficult challenge in \gls{sama} tracking. Since there is no time adherence to a metronome, tempo drifts are common and lead to small shifts in the sama instants.  

For estimating the components of meter from audio, we need signal descriptors that can be used to reliably infer the underlying meter from the surface rhythm in audio. The availability of such descriptors will greatly enhance the performance of automatic annotation algorithms. At present, we have suitable audio descriptors for low level rhythmic events such as note onsets and percussion strokes, but better descriptors for higher level rhythmic events are necessary.

The inadequate performance of the presented approaches leads us to explore the specific problems more comprehensively. It also motivates us to explore varied and unconventional approaches to rhythm analysis. Though we considered beat tracking, cycle length estimation and downbeat tracking as separate independent tasks, it might be better to consider a holistic approach and build a framework of methods in which the performance of each element can be influenced by estimations in another method. Ironically, we see from the \acrshort{HOC} algorithm, a downbeat detector for electronic dance music, that sometimes the most rigid specialization leads to good performance on apparently completely different music. Thus, it still remains an open question if we need more specialist approaches, or more general approaches that are able to react to a large variety of music. Generally, it appears desirable to have generic approaches, which can be adapted to a target music using machine learning methods that can adapt flexibly to the underlying rhythmic structures. 