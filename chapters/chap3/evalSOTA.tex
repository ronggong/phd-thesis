To conclude the chapter, we present an evaluation of the performance of some existing approaches in \gls{MIR} applied to automatic rhythm annotation tasks in Indian art music. Most of the content of this section comes from the paper by \citeA{ajay:14:rhythmJNMR}. The evaluation presented here is an early evaluation of the algorithms, and the goal of such an evaluation is not to compare performance of these algorithms with the proposed approaches. The goal is to obtain insights into the nature of rhythm in these cultures and the challenges to rhythm analysis, and to learn about the capabilities and limitations of the existing approaches when applied to Indian art music to further use these insights in proposing novel approaches. 

Many of these approaches were not proposed to handle the rhythmic structures encountered in Indian art music, and hence their performance is at best sub-optimal. The algorithms and the data had to be adapted to a common ground in which an evaluation could be done. Hence, the evaluations are not strict and comprehensive, but still provide insights into the approaches. 

We focus on the problems that are not explicitly addressed in subsequent chapters. In specific, meter estimation (cycle length estimation) and downbeat tracking are evaluated here. These two tasks however are implicitly addressed within the task of meter inference in \chapref{chap:meterInfTrack}. Cycle length estimation task is used as a proxy for \gls{tala} recognition. Downbeat tracking is an important focus of this dissertation, but we approach it together as a part of meter analysis, while the approaches evaluated here attempt downbeat tracking as an independent task. 

If existing methods from \gls{MIR} are capable of handling the following tasks in a satisfying way for Indian art music, we will be able to automatically analyze the content of these music signals in a well-structured way. However, as recent research results show \cite{holzapfel:12:beat}, these tasks are far from being solved even for the Eurogenetic forms of music, for which most methods have been presented. We evaluate several approaches for each of the three tasks and analyze which of those are promising in their results and can provide directions for future work. It should be pointed out here that there are algorithmic approaches which tackle more than one task in a combined way \cite[e.g.,]{klapuri:06:meter}. We will report the accuracy of individual tasks for such systems in our experiments as well. Further, it is also to be noted that we use only audio and its associated metadata in these tasks, because none of the available methods are capable of combining audio processing with the several other cues that were specified in \secref{sec:probdef:opportunities}.
\subsubsection{Datasets for evaluation}
The recordings used for evaluation in this section are a subset of the bigger CompMusic collection that is described in detail in \chapref{chap:datasets}. The CompMusic collection is a comprehensive collection representative of Indian art music, and in the context of this section, we use only a subset of the audio collection and the associated rhythm metadata from commercially available releases. The audio recordings are short clips extracted from full length pieces. 

In order to evaluate the algorithms, we need collections of audio recordings that are annotated in various aspects. For cycle length recognition we only need high-level information about the \gls{tala}, which decides the length of the \gls{tala}. For the tasks of downbeat tracking however, we need low-level annotations that specify the alignment between organized pulsation and music sample. Because no such annotated music collection was available, a collection of samples had to be manually annotated. As the process of manual annotation is very time consuming, we decided to compile a bigger set of recordings with high-level annotation and selected a smaller set of recordings for the evaluation and downbeat tracking. 

For both Hindustani and Carnatic music, recordings from four popular \glspl{tala} were selected for evaluation. The Carnatic dataset has 61, 63, 60, and 33 pieces in \gls{adi}, \gls{rupaka}, \gls{mishra chapu}, and \gls{khanda chapu} \glspl{tala}, respectively. The Hindustani dataset has 62, 61, 19, and 15 pieces in \gls{teental}, \gls{ektal}, \gls{jhaptal}, and \gls{rupak}, respectively. The Hindustani dataset has compositions in three \gls{lay} classes - \gls{vilambit}, \gls{madhyam} and \gls{dhrut}. In the datasets, the pieces are 2 minute long excerpts sampled at 44100 Hz. Though the audio recordings are stereo, they are down-mixed to mono since none of the algorithms evaluated in this study make use of additional information from stereo audio and primarily work on mono. They include instrumental as well as vocal recordings. The \gls{tala}/\gls{taal} annotation of these pieces were directly obtained from the accompanying editorial metadata contained in the CompMusic collection. 

The downbeat recognition task is evaluated only on Carnatic music, with piecess from \gls{adi} and \gls{rupaka} \gls{tala}. Thirty two examples in \gls{adi} \gls{tala} and thirty four examples in \gls{rupaka} \gls{tala} of Carnatic music have beat and sama instants manually annotated, which we refer to as the Carnatic low-level-annotated dataset. Similar to the Carnatic dataset, Carnatic low-level-annotated dataset also consists of two minute long excerpts. All annotations were manually done using Sonic visualizer~\cite{cannam:10:sv} by tapping along to a piece and then manually correcting the annotations. 
\subsection{Cycle length estimation}
The algorithms that are evaluated for cycle length estimation can be divided into two substantially different categories. On one hand, we have approaches that examine the characteristics in the surface rhythm of a piece of music, and try to derive an estimate of cycle length solely based on the pulsations found in that specific piece - called self-contained approaches in this section. The self-contained approaches evaluated are:
\begin{description}[leftmargin=*]
	\item[\acrshort{GUL} algorithm:]The meter estimation algorithm by \citeA{sankalp:12:meter} that focused on music with a regular divisive meter. Further, the algorithm only considers a classification into double, triple, or a septuple meter. Therefore, we had to restrict the evaluation to those classes that are based on such a meter. For Carnatic music, \gls{adi}, \gls{rupaka}, and \gls{mishra chapu} \glspl{tala} have a double, triple and septuple meter respectively. In Hindustani music, \gls{teental}, \gls{ektal}, and \gls{rupak} were annotated to belong to double, triple and septuple meter classes. 
	\item[\acrshort{PIK} algorithm:]The time signature estimation algorithm proposed by \citeA{pikrakis:04:meter}. The approach presents two different diagonal processing techniques and we report the performance for both methods (Method-A and Method-B). As suggested by \citeauthor{pikrakis:04:meter}, we also report the performance using a combination of the two methods. 
	\item[\acrshort{KLA} algorithm:]The meter analysis algorithm proposed by \citeA{klapuri:06:meter} can be used for cycle length recognition task by using the bar, beat, and subdivision interval durations. Ideally, dividing the inter-downbeat interval by the inter-beat interval should present us with the bar length in beats. However, we explore the use of the bar-beat, beat-subdivision, and bar-subdivision interval relations to estimate cycle length and evaluate how well they coincide with the known cycle lengths of a piece. 
	\item[\acrshort{SRI} algorithm:]Similar to \acrshort{KLA} algorithm, the long term periodicity and the sub-beat structure estimated by the algorithm proposed by \citeA{ajay:12:beatWkShop} can be used for cycle length recognition, and we explore the use of the bar-beat, beat-subdivision, and bar-subdivision interval relations to estimate cycle length. For the present evaluation, the tempo estimation in the algorithm, which is adapted from \citeA{davies:07:beat}, is modified to peak at 90 \bpm. Further, the tempo analysis was modified to include a wide range of tempi (from 20 \bpm\ to 180 \bpm). 
\end{description}
On the other hand, there are rhythmic similarity approaches that can give an insight into the rhythmic properties of a piece by comparing with other pieces of known rhythmic content. To this end, we will use the music collections that contain pieces with known cycle lengths. There, we can determine the rhythmic similarity of an unknown piece to all the pieces in our collection. We can then assign a cycle length to the piece according to the observation of the cycle lengths of other similar pieces. The approaches based on rhythm similarity measures (called Comparative approaches in this section) evaluated are: 
\begin{description}[leftmargin=*]
  \item[\acrshort{OP} algorithm:]The approach proposed by \citeA{pohle:09:rhythmSim} that uses Onset Patterns (OP) as the rhythm similarity measure.
	\item[\acrshort{STM} algorithm:]The approach proposed by \citeA{holzapfel:10:scale} that uses Scale Transform Magnitudes (STM) as the rhythm similarity measure.
\end{description}

\subsubsection{Evaluation criteria} 
For comparative approaches, we apply a 1-nearest-neighbor classification in a leave-one-out scheme, and report the accuracy for a dataset. For self-contained approaches, we examine the accuracy of the outputs obtained from various algorithms. 

We note that the algorithms \acrshort{PIK} and \acrshort{GUL} consider short time scales for cycle lengths and may track cycles of shorter length than the measure cycle. Hence, as explained in \secref{sec:probdef:challenges}, the algorithms may track meter at the subdivision level. As the algorithms were not specifically designed to perform the task of cycle length recognition as defined in \secref{sec:probdef:autoannot}, the evaluation has to be adapted to the algorithms. For example, \acrshort{GUL} classifies the audio piece into three classes - duple, triple, and septuple meter. For this reason, samples in the dataset are labeled as being duple, triple or septuple based on the \gls{tala} for evaluating \acrshort{GUL}. Rhythm classes in the datasets that do not belong to any of these categories are excluded from evaluation. 

We are primarily interested in estimating the cycle length at the \gls{avart}/\gls{avartana} level, a problem related to estimating the measure length in Eurogenetic music. However, as explained in \secref{sec:probdef:challenges}, cycles may exist at several metrical levels, with especially Carnatic \glspl{tala} having equal subdivisions at lower metrical levels in many cases. In connection with the fact that the measure cycles might extend over a long period of time, these shorter cycles contribute an important aspect to forming what can be perceived as beats. For the evaluations on Carnatic music in this section, we will refer to the subdivision meter and the cycle length as given in \tabref{tab:cm:talastruct}. Since there is no well-defined subdivision meter in Hindustani music, we will refer to only the cycle length in number of \glspl{matra} from \tabref{tab:hm:taalstruct}.

For \acrshort{KLA} and \acrshort{SRI} algorithms we report the accuracy of estimating the annotated cycle length at the \gls{CML}. We also report the \gls{AML} accuracy considering cycle length estimates by the algorithms to be correct that are related to the annotated cycle length by a factor of 2 or 1/2, which is referred to as doubling or halving, respectively. For cycle lengths which are odd we only consider doubling of cycle length estimates in \gls{AML}. Halving and doubling of cycle lengths can be interpreted as estimating sub-cycles and supra-cycles related to the annotated cycle length by a multiple, and can provide insights on tempo estimation errors committed by the algorithms. Though the \gls{tala} cycle is an important part of rhythmic organization, it is not necessary that all phrase changes occur on the \gls{sama}. In \gls{adi} \gls{tala} for example, most of the phrase changes occur at the end of the 8 beat cycle, there are compositions where some phrase changes and strong accents occur at the end of half-cycle or the phrase might span over two cycles (16 beats). Hence, in this case a cycle length of 4, 8, or 16 would be acceptable, depending on the composition. This needs to be considered when we evaluate the performance of algorithms. 

\subsubsection{Self-contained approaches}
We differentiate between self-contained and comparative approaches, and the self-contained approaches are divided into two types of methods. The first type attempts to estimate the meter or the time signature based on repetitions observed in the signals, while the second type aims at tracking the pulsations related to those repetitions. We start our evaluations with methods that belong to the first type (\acrshort{GUL}, \acrshort{PIK}), and evaluate then the tracking methods (\acrshort{KLA}, \acrshort{SRI}). 
\begin{table}
\centering
\begin{tabular}{@{}lr@{}}\toprule
Dataset & Accuracy (\%)  					    \tabularnewline \midrule
Carnatic (without \gls{khanda chapu}) & 75.27  \tabularnewline
Hindustani (without \gls{jhaptal}) & 49.30  	\tabularnewline \bottomrule
\end{tabular}
\caption{Performance of meter estimation using \protect\acrshort{GUL} algorithm.}\label{tab:jnmreval:cycleGUL} 
\end{table} 

\tabref{tab:jnmreval:cycleGUL} shows the accuracies for the two datasets, using the types of rhythms that can be processed by the algorithm. The performance on Carnatic music is better than the performance on Hindustani music. A detailed analysis revealed that the performance on \gls{rupaka} \gls{tala} is only 65.08\%, which leads to considerable decrease in the performance on Carnatic music. This poorer performance can be attributed to the ambiguity between duple and triple meter that is an intrinsic property of this \gls{tala} (see \secref{sec:probdef:challenges}). Furthermore, the performance on Hindustani music was found to be poor on \gls{rupak} and \gls{ektal} while the performance on just \gls{teental} is 80.64\%. This can be attributed to the fact that there are very long cycles in Hindustani music in \gls{vilambit} \gls{lay}, where the long subdivision time-spans restrains the algorithm from a correct estimation. In most of such cases in \gls{ektal} and \gls{rupak}, the estimated meter is a duple meter, which might be related to the further division of the \glspl{matra} using filler strokes. 
\begin{table}
\centering
\begin{tabular}{@{}lrrr@{}}\toprule
Dataset & Method-A & Method-B & Combined\tabularnewline \midrule
Carnatic & 52.53 & 49.30 & 64.06	\tabularnewline
Hindustani & 35.67 & 53.50 & 57.96	\tabularnewline \bottomrule
\end{tabular}
\caption[Performance of cycle length estimation using \protect\acrshort{PIK} algorithm]{Performance of cycle length estimation using \protect\acrshort{PIK} algorithm. The Method-A and Method-B refer to the two methods suggested by \protect\citeA{pikrakis:04:meter}. All values are in percentage.}\label{tab:jnmreval:cyclePIK} 
\end{table}

Pikrakis algorithm (\acrshort{PIK}) looks for measure lengths between 2 and 12. We report the accuracy accepting an answer if it is correct at one of the metrical levels. For example, for \gls{adi} \gls{tala} and \gls{teental}, 4/4, 8/4, 4/8, 8/8 are all evaluated to be correct estimates, because 4 is the subdivision meter, and 8 is the length of the \gls{avartana} (cycle length). Further, the algorithm outputs an estimation for every 5 second frame of audio, and therefore time signature of a song is obtained by using a majority vote for a whole song. The performance is reported as the accuracy of estimation (\% correctly estimated) for both the diagonal processing methods (Method-A and Method-B) in \tabref{tab:jnmreval:cyclePIK}. As suggested by \citeauthor{pikrakis:04:meter}, we also use both methods to combine the decision and it improves the performance, as can be seen from the table. The performance on Carnatic music is better than that on Hindustani music. Though the performance on Hindustani dataset is poor, further analysis shows that for \gls{teental}, the accuracy is 74.19\%. \acrshort{PIK} algorithm performs better in the cases where the meter is a simple duple or triple, while the performance is worse with other meters. For example, \gls{mishra chapu} (length 7) has an additive meter and the cycle can be visualized to be a combination of 3/4 and 4/4. On that class the \acrshort{PIK} algorithm estimates most of \gls{mishra chapu} pieces to have either a 3/4 meter or a 4/4 meter. 

To evaluate the tracking methods, we can compare the pulsations estimated by the algorithms with the ground truth annotations at all three metrical levels to determine if the large possible tempo ranges cause the beat to be tracked at different levels of the meter. From the estimates obtained from \acrshort{KLA} for downbeats, beats and subdivision pulses on a specific piece, we define the following time-spans: let $T_c$ denote the median cycle duration (inter-downbeat interval), $T_b$ the median beat duration, and $T_a$ the median subdivision duration. We use a different terminology for these as compared to $\isi$, $\ibi$, and $\iai$ defined in \secref{sec:probdef:thesismeter} to highlight the difference that these approaches evaluated here were not specifically designed for Indian art music. We then compute the cycle length estimates as, 
\begin{eqnarray}
L_{cb} = \left\lfloor \frac{T_c}{T_b} \right\rceil \nonumber \;\;
L_{ca} = \left\lfloor \frac{T_c}{T_a} \right\rceil \nonumber \;\;
L_{ba} = \left\lfloor \frac{T_b}{T_a} \right\rceil \nonumber
\end{eqnarray}
where $\lfloor.\rceil$ indicates rounding to the nearest integer. We examine which of the three estimates more closely represents the cycle length. We report both the \gls{CML} and \gls{AML} accuracy of cycle length recognition. \tabref{tab:jnmreval:cycleKLA} shows the recognition accuracy (in percentage) of \acrshort{KLA} algorithm separately for $L_{cb}$, $L_{ca}$, or $L_{ba}$ as the cycle length estimates. 
%
\begin{table}
\centering
\begin{tabular}{@{}lcccccccc@{}} \toprule
 & & \multicolumn{3}{c}{\acrshort{CML} (\%)} & & \multicolumn{3}{c}{\acrshort{AML} (\%)}\tabularnewline 
Dataset & & $L_{cb}$ & $L_{ca}$ & $L_{ba}$ & & $L_{cb}$ & $L_{ca}$ & $L_{ba}$ \tabularnewline \midrule
Carnatic & & 11.06 & 8.76 & 4.15 & & 34.10 & 45.16 & 25.81 \tabularnewline \addlinespace[3pt]
Hindustani & & 0.00 & 25.40 & - & & 45.22 & 46.50 & - \tabularnewline \bottomrule
\end{tabular}
\caption[Accuracy of cycle length recognition using \acrshort{KLA} algorithm]{Accuracy of cycle length recognition using \acrshort{KLA} algorithm. Subdivision meter ($L_{ba}$) in Hindustani music is not well-defined and hence omitted.}\label{tab:jnmreval:cycleKLA}
\end{table}
%
\begin{table}
\centering
\begin{tabular}{@{}lcccccccc@{}} \toprule
 & & \multicolumn{3}{c}{\acrshort{CML} (\%)} & & \multicolumn{3}{c}{\acrshort{AML} (\%)}\tabularnewline
Dataset & & $L_{cb}$ & $L_{ca}$ & $L_{ba}$ & & $L_{cb}$ & $L_{ca}$ & $L_{ba}$ \tabularnewline \midrule
Carnatic & & 3.69 & 0.46 & 6.45 & & 40.55 & 50.69 & 14.28 \tabularnewline \addlinespace[3pt]
Hindustani & & 14.64 & 9.55 & - & & 43.95 & 55.41 & - \tabularnewline \bottomrule
\end{tabular}
\caption[Accuracy of cycle length recognition using \acrshort{SRI} algorithm]{Accuracy of cycle length recognition using \acrshort{SRI} algorithm. Subdivision meter ($L_{ba}$) in Hindustani music is not well-defined and hence omitted.}
\label{tab:jnmreval:cycleSRI}
\end{table}

We see in \tabref{tab:jnmreval:cycleKLA} that there is a large difference between \gls{CML} and \gls{AML} performance, which indicates that in many cases tracked level is related to the annotated level by a factor 2 or 1/2. We also see that for Hindustani music, the cycle length is best estimated using $L_{ca}$, with the \gls{CML} accuracy being very low or zero when we use the other cycle length estimates instead. As discussed earlier, in Hindustani music, the cycle length is defined as the number of \glspl{matra} in the cycle. However, in the case of \gls{vilambit} pieces, the \glspl{matra} are longer than the range of the tatum pulse time-span estimated by the algorithm and hence the performance is poor. Interestingly, we see a good performance when evaluated with $L_{cb}$ only with \gls{teental}, which resembles the Eurogenetic 4/4 meter, with an \gls{AML} accuracy of 88.71\% in spite of the \gls{CML} accuracy being zero. In fact, it is seen that $L_{cb}$ is always four in the case of a correct estimation (\gls{AML}), which is the estimate of the number of \glspl{vibhaag} in the \gls{taal}. Further, it follows from \citeA[Figure 8]{klapuri:06:meter} that relation between neighboring levels in \acrshort{KLA} cannot be larger than 9, which implies longer cycle length estimates (as needed by e.g. \gls{ektal} or \gls{teental}) could possibly appear only in the $L_{ca}$ length. 

The \gls{CML} accuracy in Carnatic dataset with $L_{cb}$ is better than the other cycle length estimates, showing that \acrshort{KLA} tracked correct tempo in a majority of cases in Carnatic music. However, the performance is poor because the algorithm often under-estimates the cycle length. Further, in \glspl{tala} of Carnatic music that have two \glspl{akshara} in a beat (\gls{khanda chapu} and \gls{mishra chapu}), $L_{ca}$ is a better indicator of the cycle length than $L_{cb}$, since \glspl{akshara} are closer to the estimated subdivision duration. In general, $L_{ba}$ performs poorly compared to $L_{ca}$ or $L_{cb}$, which is not astonishing since the cycle lengths we are looking for are longer than the estimated subdivision meter. Summing up, none of the estimated meter relations can serve as a robust estimate for the \gls{avartana} cycle length. 

\acrshort{SRI} algorithm estimates the cycle length at two metrical levels using the beats tracked by \citeA{ellis:07:beat} beat tracker, one being at the cycle level (bar length in beats), and the second at the beat level (subdivision meter, or \gls{nade}). The algorithm computes a list of possible candidates for the subdivision meter and bar length, ordered by a score. We consider the top candidate in the list and compute the cycle length estimates $L_{cb}$, $L_{ba}$, and the $L_{ca}$, assuming that the beats tracked by Ellis beat tracker correspond the beat duration $T_b$. Similar to \acrshort{KLA} algorithm, we present the \gls{CML} and \gls{AML} accuracy of performance in \tabref{tab:jnmreval:cycleSRI}. 

We see that there is large disparity between the \gls{CML} and \gls{AML} accuracy, which indicates that the beat tracker and the correct beat are related by a factor of $2$ or $1/2$. In general, the algorithm performs poorly, which can be mainly attributed to errors in tempo and beat tracking. The tempo estimation uses a weighting curve that peaks at 90 beats per minute, which is suitable for Carnatic music, but leads to an incorrect estimation of cycle length for Hindustani music. A beat tracking based approach as the \acrshort{SRI} algorithm might in general not be well suited for Hindustani music which often includes long cycles. % that might be more reflected in the structure of melodic phrases than in pulsation and rhythmic aspects. 

The poor performance on Carnatic music can in part be also attributed to variation in percussion accompaniment, which is completely free to improvise within the framework of the \gls{tala}. Further, the algorithm is based on the implicit assumption that beats at the same position in a measure cycle are similar between various recurrences of the cycle. For certain music pieces where there are no inherent rhythmic patterns or the patterns vary unpredictably, the algorithm gives a poorer performance. For Carnatic music, the algorithm specifically estimates the subdivision-meter (\gls{nade}), as the number of \glspl{akshara} per beat. Using $L_{ba}$ as an estimate of the \gls{nade}, we obtain a reasonably good performance comparable to \acrshort{GUL} with an accuracy of 39.63\% and 79.72\% at \gls{CML} and \gls{AML} (of the subdivision meter), respectively. We see that a reasonable performance when demanding an exact numerical result for the meter (\gls{CML}) is only reached for the \gls{nade} estimation in Carnatic music.

We observe that the duration of cycles in seconds is often estimated correctly, but the presence or absence of extra beats causes the estimated length in beats to be wrong. Ellis beat tracker is sensitive to tempo value and cannot handle small tempo changes effectively. This leads to addition of beats into the cycle and the cycle length in many cases were estimated to be one-off from the actual value, though the actual duration of the cycle (in seconds) was estimated correctly. 
%
\subsubsection{Comparative approaches}
The comparative approaches are based on a description of periodicities that can be derived from the signal without the need to perform meter tracking. Performances of the two evaluated methods, \acrshort{OP} and \acrshort{STM}, is the average accuracy in a 1-nearest neighbor classification. It tells us how often a piece found to be most similar to a test piece belongs actually to the same class of rhythm as the test piece. The results of this classification experiment are depicted in \tabref{tab:jnmreval:cycleCOMP}. It is apparent that the comparative approaches lead to a performance significantly better than random, which would be 25\% for our compiled four-class datasets. In fact, accuracies are in the same range as the results of the \acrshort{PIK} algorithm, with \acrshort{PIK} performing better on Carnatic music (64.1\% instead of 42.2\%). This might indicate the potential of combining self-contained and comparative approaches, because none of the approaches evaluated for cycle length recognition provide us with a sufficient performance for a practical application. 
\begin{table}
\centering
\begin{tabular}{@{}lcr@{}} \toprule
Dataset & \acrshort{OP} (\%)  & \acrshort{STM} (\%) \tabularnewline \midrule
Carnatic & 41.0  & 42.2    \tabularnewline 
Hindustani & 47.8  & 51.6   \tabularnewline \bottomrule
\end{tabular}
\caption{Accuracy of cycle length recognition using comparative approaches}\label{tab:jnmreval:cycleCOMP} 
\end{table}
%
% \subsection{Beat Tracking}
% \subsection{Beat tracking experiment} % \label{sec:beat}
So far, to the best of our knowledge, no beat tracking has been attempted on Indian and Turkish music. Hence, we chose two acknowledged approaches presented in the last decade \cite{klapuri:06:meter,ellis:07:beat} for evaluation. 

\citeA{ellis:07:beat}: \acrshort{ELL} algorithm.

BT evaluation: For our evaluations, we can compare the pulsations estimated by the algorithms with our ground truth annotations at all three metrical levels to determine if the large possible tempo ranges cause the beat to be tracked at different levels of the meter. Using the beat ground truth annotations from both the Turkish and Carnatic low-level annotated datasets, we can define the median inter-annotation time-span as $T_a$ for each music piece. From the estimates obtained from \acrshort{KLA} for downbeats, beats and subdivision pulses on a specific piece, we define the following time-spans: let $T_m$ denote the median measure time-span, $T_b$ the median beat time-span, and $T_s$ the median subdivision time-span. Let $T_{b,\acrshort{ELL}}$ denote the median beat time-span as estimated by the \acrshort{ELL} algorithm. The median is computed to obtain an estimate that is robust to spurious deviations. 
%%%%%%%%%%%%%%%%%%%%%%%%%%%%%%%%%%%%%%%%%%%%%%%%%%%%%%%%%%%%%%%%%%%%%%%%%%%%%%%%%%%%%%%%%%%%%%%%%%%%%%%%%%%%%%%%%%%%%%%%%%%%%%%
%\begin{figure*}[htb!]
  %\centering
 %\subfloat[\acrshort{KLA}]{\label{tempHist_C_am} \includegraphics[width=0.255\columnwidth]{./fig/TempHist11.png}}
 %\subfloat[\acrshort{KLA}]{\label{tempHist_C_ab} \includegraphics[width=0.245\columnwidth]{./fig/TempHist12.png}}
 %\subfloat[\acrshort{KLA}]{\label{tempHist_C_as} \includegraphics[width=0.245\columnwidth]{./fig/TempHist13.png}}
 %\subfloat[\acrshort{ELL}]{\label{tempHist_C_ell} \includegraphics[width=0.245\columnwidth]{./fig/TempHist14.png}}\\
%\caption{Relative time-span histograms on Carnatic low-level-annotated dataset. The panels (a), (b), (c) correspond to \acrshort{KLA} algorithm. Panel (d) corresponds to the \acrshort{ELL} algorithm. The ordinate is the fraction of pieces (in percentage) and is normalized to sum to 100\%.}
%\label{fig:tempHistC}
%\end{figure*}
%\begin{figure*}[htb!]
  %\centering
 %\subfloat[\acrshort{KLA}]{\label{tempHist_T_am} \includegraphics[width=0.255\columnwidth]{./fig/TempHist21.png}}
 %\subfloat[\acrshort{KLA}]{\label{tempHist_T_ab} \includegraphics[width=0.245\columnwidth]{./fig/TempHist22.png}}
 %\subfloat[\acrshort{KLA}]{\label{tempHist_T_as} \includegraphics[width=0.245\columnwidth]{./fig/TempHist23.png}}
 %\subfloat[\acrshort{ELL}]{\label{tempHist_T_ell} \includegraphics[width=0.245\columnwidth]{./fig/TempHist24.png}}\\
%\caption{Relative time-span histograms on Turkish low-level-annotated dataset. The panels (a), (b), (c) correspond to \acrshort{KLA} algorithm. Panel (d) corresponds to the \acrshort{ELL} algorithm. The ordinate is the fraction of pieces (in percentage) and is normalized to sum to 100\%.}
%\label{fig:tempHistT}
%\end{figure*}

In Hindustani music, beats of a piece are not well defined, since it depends also on the \textit{lay}. The beats could be defined at the \gls{matra} level for slow \textit{lay} and at \gls{vibhaag} level for faster pieces and hence there exists no unambiguous definition of beats \cite[p.~91]{clayton:00:time}. Due to this ambiguity which needs a deeper exploration, we do not explore beat tracking for Hindustani music formally in this paper.

We evaluate beat tracking on the Turkish low-level-annotated dataset and the Carnatic low-level-annotated dataset introduced in Section~\ref{sec:collections:lowlevel}, using \acrshort{KLA} and \acrshort{ELL} beat tracking algorithms. Since \acrshort{KLA} algorithm estimates meter at three levels, we evaluate beat tracking performance of the algorithm with the annotated pieces at the subdivision, beat and measure level to ascertain whether the algorithm tracks the annotated beats on any of those levels. We evaluate in how far the estimated beats are accurate in terms of the estimated time-spans, and if they are correctly aligned to the phase of the annotations. We further discuss various kinds of errors observed in the beat tracking results. 

We use the different time-span definitions from Section~\ref{sec:beat:evalCriteria} to obtain histograms of estimated relative tempo for both low-level-annotated datasets, as shown in Figures~\ref{fig:tempHistC} and~\ref{fig:tempHistT}. The figure shows the histogram of the ratios $T_a/T_m$, $T_a/T_b$, $T_a/T_s$, obtained from the estimates by the \acrshort{KLA} algorithm, and $T_a/T_{b,\acrshort{ELL}}$ for the estimation by \acrshort{ELL} algorithm. A ratio of one would imply that the annotation time-span and the estimated pulse time-span matched exactly, and the pulsation is thus tracked at the annotated level. A ratio of, \textit{e.g} 2, would indicate that the estimated tempo is two times faster than the annotated one. 

From the panels (a)-(c) of Figures~\ref{fig:tempHistC} and~\ref{fig:tempHistT}, we see that in a majority of pieces in both the low-level-annotated datasets, the \acrshort{KLA} algorithm tracks the annotated beats at the beat level, as indicated by Figures (\ref{tempHist_C_ab}) and (\ref{tempHist_T_ab}). It is also clear from the figure that the measure time-spans estimated by the algorithm are at least twice as long as the annotated beat time-spans and hence we conclude that the estimated downbeats never correspond to the annotated beats in both datasets. Further, we see that the estimated subdivision time-spans are much shorter than the annotated beat time-spans. Based on these observations, we present the beat tracking results using only the estimated beats for \acrshort{KLA} algorithm. In general, we can also see that there is a tendency towards tempo under-estimation for Turkish music (Figure~(\ref{tempHist_T_ab})), and towards tempo over-estimation for Carnatic music (Figure~(\ref{tempHist_C_ab})). 
\begin{table}
\begin{centering}
\begin{tabular}{@{}lrr@{}}
\toprule
Metric & $\mathrm{\acrshort{KLA}}_{\mathrm{b}}$ & \acrshort{ELL}\tabularnewline \midrule
f-measure(\%) &  46.73 &  23.27\tabularnewline
Information Gain (bits) & 1.09 &  0.4254\tabularnewline
CMLt(\%) & 34.85 & 24.78\tabularnewline \bottomrule
\end{tabular}
\par\end{centering}
\caption{Beat Tracking Results for Carnatic music, $\mathrm{\acrshort{KLA}}_{\mathrm{b}}$ indicates that the results are reported using the beat pulsation estimates of \acrshort{KLA} algorithm}\label{tab:jnmreval:btKLA}
\end{table}
%%%

The beat tracking performance is shown in Table~\ref{tbl:BeatTrack}, averaged over the songs of the individual collections. In the table, $\mathrm{\acrshort{KLA}}_{\mathrm{b}}$ refers to evaluating using the beat estimates by the \acrshort{KLA} algorithm. \acrshort{ELL} refers to the performance using Ellis beat tracking algorithm. $\mathrm{\acrshort{KLA}}_{\mathrm{b}}$ provides us with the best accuracy while beat tracking on Turkish music seems to work slightly better than for Carnatic music. However, accuracies are low in general, with \textit{e.g.} CMLt accuracies of 44.01\% and 34.85\% using $\mathrm{\acrshort{KLA}}_{\mathrm{b}}$ on Turkish and Carnatic low-level-annotated datasets, respectively. This indicates that we are able to obtain an absolutely identical sequence in less than half of the cases on both datasets. Indeed, only 33\% of the samples in the Carnatic dataset, and 29\% of the samples in Turkish dataset have a CMLt higher than 50\%, which means that only a third of the pieces are absolutely correctly tracked at least in more than half of their duration. Information gain values are very low especially for Carnatic, which indicates a lot of sequences occur that are not simply related to the annotation (e.g. by tempo halving/doubling, or off-beat). In fact, only 21\% of Carnatic, and 34\% of Turkish low-level-annotated datasets have an Information Gain value larger than 1.5 bits. This value was established as a threshold for perceptually acceptable beat tracking~\cite{zapata:12:beat}. Even though this value was established on Eurogenetic popular music, the low values for Information Gain in Table~\ref{tbl:BeatTrack} imply that a large part of the beat tracking results is not perceptually acceptable. This seems to stand in conflict with the observation made in Figures~\ref{fig:tempHistC} and~\ref{fig:tempHistT} that in a majority of cases \acrshort{KLA} seems to get a median tempo that is related to the annotation. We observed however that many beat estimations suffer from unstable phase and tempo, which causes low values especially for Information Gain and CMLt. 

\acrshort{ELL} algorithm uses the tempo estimated by \acrshort{SRI} approach \cite{ajay:12:beatWkShop}, and uses a tempo weighting function which peaks at 90 bpm. Since the median inter-annotation time-span in Turkish low-level-annotated dataset corresponds to 147 bpm, the tempo estimate is not accurate and hence the CMLt performance is poor. The median inter-annotation time-span in Carnatic dataset corresponds to 85 bpm, which is closer to the peak of the weighting function than for Turkish music. We can also see this from Figures~(\ref{tempHist_C_ell}) and (\ref{tempHist_T_ell}), where we observe that \acrshort{ELL} tracks the tempo of the annotated beats in a majority of cases in Carnatic, while a majority of pieces in the Turkish dataset are tracked with slower tempo compared to the annotations. In many cases, the estimated beats tended to drift away from the annotations. This is initially caused by the resolution of the tempo estimate (about 11.6 ms, as used by \citeA{davies:07:beat}), which can result in small deviations from the true tempo. While such errors should be compensated for in the \acrshort{ELL} algorithm by the matching of the beat estimations to the accent signal, this compensation seems to fail in many cases especially for Carnatic music. 

In order to check if beat tracking tends to happen on the off-beat for some songs in the collections, we took advantage of the properties of the evaluation measures (see Section~\ref{sec:models:bt}). Since estimations with a constant phase misalignment are practically the only reason for the f-measure to take values close to zero, we located the appearance of those songs. Interestingly, for none of the Turkish music samples such a case appeared which shows that at least in the small available dataset there is no ambiguity regarding phase alignment. However, for Carnatic music, phase alignment caused problems for both \acrshort{ELL} and \acrshort{KLA}, revealed by a large number of samples with very low f-measure (<15\%). This effect is partly related to a non-zero \gls{edupu} (a phase shift of the start of the composition relative to the sama - see Section~\ref{sec:problemdef:bg}). In such a case, the composition starts not on the sama but 2, or 4, or 6 \glspl{akshara} after \gls{sama}. If the \gls{edupu} is 2 or 6 \glspl{akshara}, the accents of the composition in the piece are more likely to be on the off-beat and hence the beat tracker tracks the off-beat, with the tracked beat instant lying half way in between the two annotated beat instants. In general, we find poorer performance with pieces with a non-zero \gls{edupu}. In Carnatic dataset we have a total of 11 pieces with a non-zero \gls{edupu}, and the average CMLt accuracy values using $\mathrm{\acrshort{KLA}}_{\mathrm{b}}$ for those pieces was observed to be clearly lower than the average (CMLt: 20.19\% instead of 34.85\%). 
% (AH: Careful, this paragraph uses number of estimated beats, the previous inter beat time-spans. These two are highly correlated. If we do not use a graphical representation to show tempo relations, we should avoid this duality.)
%
\subsection{Downbeat tracking}
% Moved to chap 2: The methods described in this section were developed for the identification of downbeats within sequences of beats. 
So far mainly music with a $4/4$ time signature was focused upon in evaluations, usually in the form of collections of Eurogenetic popular and/or classical music. Hence, we will address the questions if such approaches can cope with the lengths of cycles present in our data and if Indian art music poses challenges of unequal difficulty. The approaches evaluated are: 
\begin{description}[leftmargin=*]
\item[\acrshort{DAV} algorithm:]The algorithm proposed by \citeA{davies:06:downbeat} that assumes that percussive events and harmonic changes tend to be correlated with the downbeat.
\item[\acrshort{HOC} algorithm:]The algorithm proposed by \citeA{hockman:12:downbeat} for downbeat tracking in hardcore, jungle, and drum and bass genres of music.
\end{description}
It is apparent that both systems are conceptualized for styles of music with notable differences to Indian art music. The system by \citeA{davies:06:downbeat} is mainly sensitive to harmonic changes, whereas Indian art music does not incorporate a notion of harmony similar to the Eurogenetic concept of functional harmony. On the other hand, the system by \citeA{hockman:12:downbeat} is customized to detect the bass kick on a downbeat, which will not occur in the music we investigate here. As the latter system contains this low-frequency feature as a separate module, we will examine the influence of the low-frequency onsets and the regression separately our experiments.

\subsubsection{Evaluation results}
The evaluation metrics we use are the same as the continuity-based approach applied by \citeA{hockman:12:downbeat}. This measure applies a tolerance window of 6.25\% of the inter-annotation-interval to the annotations. Then it accepts a detected downbeat as correct, if 
\begin{enumerate}[nolistsep]
 \item The detection falls into a tolerance window.
 \item The precedent detection falls into the tolerance window of the precedent annotation.
 \item The inter-beat-interval is equal to the inter-annotation-interval (accepting a deviation of the size of the tolerance window).
\end{enumerate}
%
\begin{table}
\centering
\begin{tabular}{@{}lrr@{}} \toprule
Method & \gls{adi} (8) & \gls{rupaka} (3) \tabularnewline \midrule
\acrshort{DAV} & 21.7 & 41.2    \tabularnewline
\acrshort{HOC-SVM} & 22.9 & 42.1 \tabularnewline
\acrshort{HOC} & 49.9 & 64.4 \tabularnewline \bottomrule
\end{tabular}
\caption[Accuracy of downbeat tracking in Carnatic low-level-annotated dataset]{Accuracy of downbeat tracking on Carnatic low-level-annotated dataset. The cycle lengths are indicated in parentheses next to the \gls{tala}. All values are in percentage.}\label{tab:jnmreval:dbres} 
\end{table}
In \tabref{tab:jnmreval:dbres}, we depict the downbeat recognition accuracies (in percentage) for the two systems. The results are given separately for each of the two \glspl{tala} in the Carnatic low-level-annotated dataset. The \acrshort{HOC} algorithm was applied with and without emphasizing the low-frequency onsets, denoted as \acrshort{HOC} and \acrshort{HOC-SVM}, respectively. The \acrshort{DAV} algorithm has the lowest accuracies for all presented \glspl{tala}. This is caused by the focus of the method on changes in harmony that is related to chord changes - concepts not present in Indian art music. However, the results obtained from \acrshort{HOC} are more accurate and allows for an interesting conclusions that taking onsets in the low-frequency region into account improves recognition for all contained rhythms. However, Carnatic music with its wide rhythmic variations and its flexible rhythmical style seems to represent a more difficult challenge for downbeat recognition, with the range of accuracy smaller than that reported for electronic dance music \cite{hockman:12:downbeat}. Pieces without such phenomenal cues are very likely to present both automatic systems and human listeners with a more difficult challenge when looking for the downbeat. Furthermore, the accuracies depicted in \tabref{tab:jnmreval:dbres} can only be achieved with known cycle length, and correctly annotated beats, tasks that are already complex and cannot be achieved with a high accuracy. 
%
\subsection{Discussion}
We summarize and discuss the key results of the evaluation presented in this section. The results provide us with useful insights to indicate promising directions for further work. At the outset, the results indicate that the performance of evaluated approaches is not adequate for the presented tasks, and that methods that are suitable to tackle the culture specific challenges in computational analysis of rhythm need to be developed. 

Cycle length estimation is challenging in Indian art music since cycles of different lengths exist at different time-scales. Although we defined the most important cycle to be at the \gls{avart}/\gls{avartana} level, the other cycles, mainly at the beat and subdivision level, also provide useful rhythm related information. The evaluated approaches \acrshort{PIK} and \acrshort{GUL} estimate the subdivision meter and time signature. This is possible to an acceptable level of accuracy, when restricting to a subset of rhythm classes with relatively simple subdivision meters. Though they do not provide a complete picture of the meter, they estimate the underlying metrical structures at short time scales and can be used as pre-processing steps for estimating longer and more complex cycles. 

Both \acrshort{SRI} and \acrshort{KLA} aimed to estimate the longer cycle lengths but show a performance that is inadequate for any practical application involving cycle length estimation. Tempo estimation and beat tracking have a significant effect on cycle length estimation, especially in the self-contained approaches and also need to be explored further. The comparative approaches show that the applied signal features capture important aspects of rhythm but are not sufficient to be used standalone for cycle estimation. A combination of self-contained and comparative approaches might provide useful insights into rhythm description of Indian art music through mutual reinforcement. %Developing methods that take the culture-specific properties of rhythm into account is therefore necessary to proceed towards a reliable computational rhythm analysis.

Downbeat tracking was explored using \acrshort{HOC} and \acrshort{DAV} algorithms. The downbeat detectors evaluated here needed an estimation of beats and the cycle length of the piece, which in themselves are difficult to estimate. Since downbeat information can help in estimating the cycle length and also beat tracking, a joint estimation of the beat, cycle length, and downbeat might be a potential solution since each of these parameters are mutually useful for estimating the others. A combination of bottom up and top down knowledge based approach which performs a joint estimation of these parameters is to be explored further, using models that better represent the underlying metrical structures. 

Long \gls{avart} cycle is a significant challenge in Hindustani music. For \glspl{taal} with a very long cycle duration, estimating the correct metrical level is essential and methods that aim at tracking short time-span pulsation will be not adequate due to the grouping structure of the \gls{taal}. With a wide variety of rhythms, coupled with the perceptual \gls{edupu}, Carnatic music poses a difficult challenge in \gls{sama} tracking. Since there is no time adherence to a metronome, tempo drifts are common and lead to small shifts in the sama instants.  

For estimating the components of meter from audio, we need signal descriptors that can be used to reliably infer the underlying meter from the surface rhythm in audio. The availability of such descriptors will greatly enhance the performance of automatic annotation algorithms. At present, we have suitable audio descriptors for low level rhythmic events such as note onsets and percussion strokes, but better descriptors for higher level rhythmic events are necessary.

The inadequate performance of the presented approaches leads us to explore the specific problems more comprehensively. It also motivates us to explore varied and unconventional approaches to rhythm analysis. Though we considered beat tracking, cycle length estimation and downbeat tracking as separate independent tasks, it might be better to consider a holistic approach and build a framework of methods in which the performance of each element can be influenced by estimations in another method. Ironically, we see from the \acrshort{HOC} algorithm, a downbeat detector for electronic dance music, that sometimes the most rigid specialization leads to good performance on apparently completely different music. Thus, it still remains an open question if we need more specialist approaches, or more general approaches that are able to react to a large variety of music. Generally, it appears desirable to have generic approaches, which can be adapted to a target music using machine learning methods that can adapt flexibly to the underlying rhythmic structures. 
%
% Preliminary experiments~\cite{Ajay:CompMusic21} for tāla recognition using beat tracking emphasize the need for a culture-specific approach. We developed an algorithm that uses a beat similarity matrix and inter onset interval histogram to automatically extract the sub-beat structure and the supra-beat periodicity of a musical piece. From this information, we could obtain a rank ordered set of candidates for the tāla cycle period and the naḍe. A block diagram of the system is in Figure \ref{fig:FullBD}. The algorithm was tested on a manually annotated Carnatic music dataset (CMDB) consisting of 86 thirty second song snippets of both vocal and instrumental music with different instrumentation, set to different tālas and naḍe. The algorithm was also tested on an Indian light classical music dataset (ILCMDB) consisting of 58 semi-classical songs based on popular Hindustani \emph{ragas}. The allowed metrical level (AML)~\cite{Davies:07} recognition accuracy of the algorithm on ILCMDB was 79.3\% and 72.4\% for the naḍe and the tāla, respectively. The accuracy on the more difficult CMDB was poorer with 68.6\% and 51.1\% for naḍe and tāla, respectively. The poorer performance on CMDB can be attributed to changes in kāla (metrical level) through the song and the lack of distinct beat-level similarity in the songs of the dataset. This is quite typical in Carnatic music where the percussion accompaniment is completely free to improvise within the framework of the tāla. The performance was also poor on the songs in odd beat tālas such as Mishra Chapu and Khanda Chapu. This motivates us further to explore knowledge based approaches to tāla recognition.
%
%
% Recently, our focus mainly has been on sama estimation. The information about the sama instants of a music piece makes the other annotations tasks much simpler. The strokes of Mridangam and Tabla are quite useful for annotation tasks. We are thus exploring the use of percussion enhancement algorithms~\cite{Fitzgerald2010} on audio to obtain only percussive onsets during onset detection. Many of the phrase changes occur at the sama instants and are characterized by notable melodic, percussive, and timbral changes. However, neither of each of these changes are necessary nor sufficient indicators of sama. This motivates us to explore a hybrid approach involving the information from onsets, melody, and timbre for sama estimation (Figure \ref{fig:SamaEstimate}). For melody and timbre change point estimation, novelty functions~\cite{Foote2000a} estimated using HPCP~\cite{emilia:phd} and MFCC provide promising results. Further experimentation towards sama recognition is under progress. 
%%%%%%%%%%%%%%%%%%%%%%%%%%%%%%%%%%%%%%%%%%%%%% THESIS PROPOSAL END %%%%%%%%%%%%%%%%%%%%%%%%%%%%%%%%%%%%%%%%%%%%%%%%%%%%
%
% A well-formed metrical grid is often assumed to be present in Eurogenetic tonal music~\cite{lerdahl:83:generative}. Forms of music that deviate from the well-formedness in terms of meter have rarely been in the focus of computational rhythm analysis (\textit{e.g.} the work by \citeA{Antonopoulos07}). In this paper, we concentrate our studies on Carnatic and Hindustani music of India, as well as Makam music of Turkey. All the three musics have a long standing history within the respective cultures. They are also well established traditions that exist in current social context with a large audience and significant musicological literature . While each of these music traditions includes a huge variety of styles and forms in itself, we maintain a broader point of view here, and ask in how far the state of the art in rhythm analysis can present us with meaningful insights when applied to them. 
%
% Until now, in the context of Hindustani music, approaches for tempo estimation and time signature recognition were presented by Gulati et al.~\cite{gulati:10:pattern,sankalp:12:meter}, and transcription of percussive instruments was approached by \citeA{chordia:05:phdthesis} and \citeA{miron:11:thesis}. For Carnatic music, \citeA{ajay:12:beatWkShop} proposed a system to describe meter in terms of the time-span relations between pulsations at measure, beat and subdivision levels. An approach to discriminate between classes of rhythm in Turkish music was proposed by \citeA{Holzapfel09ISMIR}, which can serve as a tool for time signature recognition as well. Recently, \citeA{holzapfel:12:meterTurkWkShop} investigated how the melodies of Turkish makam music are related with the underlying rhythmic mode. Both mentioned approaches for Turkish music examined repertoire consisting of symbolic data, and to the best of our knowledge no rhythmic analysis of audio signals has been attempted for Turkish music. 
%
%\comment{The available descriptions of rhythm and meter in Indian and Turkish music imply that they differ from Eurogenetic music in terms of the time-spans at various levels of the meter hierarchy and in terms of the possibly irregular pulsation at one of the levels (see \citeA<e.g.>{clayton:00:time,ozkan:84:usul}). Therefore, in this article we address the problem of tracking the pulsations at different metrical levels, and of determining the relations between the time-spans between these levels. By doing so we can potentially derive meaningful descriptions of the metric structure of Indian and Turkish music pieces. In Section \ref{sec:problemdef} we provide the reader with the basic concepts related to meter in Turkish Makam, Carnatic and Hindustani music, and describe some of the opportunities and challenges to rhythm analysis in these music cultures. We will then define three analysis tasks motivated by the structure of meter in the three music cultures - beat tracking, cycle length recognition, and downbeat detection. As none of these tasks have been addressed before on the presented repertoire, we had to compile and annotate the music collections that are necessary for the evaluation of the analysis algorithms. The related collections and annotations will be described in Section~\ref{sec:collections}. In \secref{sec:models} we give a detailed description of all the analysis algorithms that we will evaluate for the three tasks along with description of the evaluation methods. The following Sections~\ref{sec:beat}-\ref{sec:downbeat} give the detailed results of our experiments, separately for each of the analysis tasks. In Section~\ref{sec:discussion} we sum up the results and describe what they imply for rhythm analysis in Indian and Turkish music. The final section specifies future directions for research in rhythm analysis.}
% \comment{In this part of the chapter, JNMR article is to be included. Explain the details of all algorithms and explain where they worked and where they failed. The dataset used is a bit obsolete and hence need not be included in detail here. May be push this in text within each problem discussion ??}