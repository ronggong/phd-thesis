The algorithms that are evaluated for cycle length estimation can be divided into two substantially different categories. On one hand, we have approaches that examine the characteristics in the surface rhythm of a piece of music, and try to derive an estimate of cycle length solely based on the pulsations found in that specific piece - called self-contained approaches in this section. The self-contained approaches evaluated are:
\begin{description}[leftmargin=*]
	\item[\acrshort{GUL} algorithm:]The meter estimation algorithm by \citeA{sankalp:12:meter} that focused on music with a regular divisive meter. Further, the algorithm only considers a classification into double, triple, or a septuple meter. Therefore, we had to restrict the evaluation to those classes that are based on such a meter. For Carnatic music, \gls{adi}, \gls{rupaka}, and \gls{mishra chapu} \glspl{tala} have a double, triple and septuple meter respectively. In Hindustani music, \gls{teental}, \gls{ektal}, and \gls{rupak} were annotated to belong to double, triple and septuple meter classes. 
	\item[\acrshort{PIK} algorithm:]The time signature estimation algorithm proposed by \citeA{pikrakis:04:meter}. The approach presents two different diagonal processing techniques and we report the performance for both methods (Method-A and Method-B). As suggested by \citeauthor{pikrakis:04:meter}, we also report the performance using a combination of the two methods. 
	\item[\acrshort{KLA} algorithm:]The meter analysis algorithm proposed by \citeA{klapuri:06:meter} can be used for cycle length recognition task by using the bar, beat, and subdivision interval durations. Ideally, dividing the inter-downbeat interval by the inter-beat interval should present us with the bar length in beats. However, we explore the use of the bar-beat, beat-subdivision, and bar-subdivision interval relations to estimate cycle length and evaluate how well they coincide with the known cycle lengths of a piece. 
	\item[\acrshort{SRI} algorithm:]Similar to \acrshort{KLA} algorithm, the long term periodicity and the sub-beat structure estimated by the algorithm proposed by \citeA{ajay:12:beatWkShop} can be used for cycle length recognition, and we explore the use of the bar-beat, beat-subdivision, and bar-subdivision interval relations to estimate cycle length. For the present evaluation, the tempo estimation in the algorithm, which is adapted from \citeA{davies:07:beat}, is modified to peak at 90 \bpm. Further, the tempo analysis was modified to include a wide range of tempi (from 20 \bpm\ to 180 \bpm). 
\end{description}
On the other hand, there are rhythmic similarity approaches that can give an insight into the rhythmic properties of a piece by comparing with other pieces of known rhythmic content. To this end, we will use the music collections that contain pieces with known cycle lengths. There, we can determine the rhythmic similarity of an unknown piece to all the pieces in our collection. We can then assign a cycle length to the piece according to the observation of the cycle lengths of other similar pieces. The approaches based on rhythm similarity measures (called Comparative approaches in this section) evaluated are: 
\begin{description}[leftmargin=*]
  \item[\acrshort{OP} algorithm:]The approach proposed by \citeA{pohle:09:rhythmSim} that uses Onset Patterns (OP) as the rhythm similarity measure.
	\item[\acrshort{STM} algorithm:]The approach proposed by \citeA{holzapfel:10:scale} that uses Scale Transform Magnitudes (STM) as the rhythm similarity measure.
\end{description}

\subsubsection{Evaluation criteria} 
For comparative approaches, we apply a 1-nearest-neighbor classification in a leave-one-out scheme, and report the accuracy for a dataset. For self-contained approaches, we examine the accuracy of the outputs obtained from various algorithms. 

We note that the algorithms \acrshort{PIK} and \acrshort{GUL} consider short time scales for cycle lengths and may track cycles of shorter length than the measure cycle. Hence, as explained in \secref{sec:probdef:challenges}, the algorithms may track meter at the subdivision level. As the algorithms were not specifically designed to perform the task of cycle length recognition as defined in \secref{sec:probdef:autoannot}, the evaluation has to be adapted to the algorithms. For example, \acrshort{GUL} classifies the audio piece into three classes - duple, triple, and septuple meter. For this reason, samples in the dataset are labeled as being duple, triple or septuple based on the \gls{tala} for evaluating \acrshort{GUL}. Rhythm classes in the datasets that do not belong to any of these categories are excluded from evaluation. 

We are primarily interested in estimating the cycle length at the \gls{avart}/\gls{avartana} level, a problem related to estimating the measure length in Eurogenetic music. However, as explained in \secref{sec:probdef:challenges}, cycles may exist at several metrical levels, with especially Carnatic \glspl{tala} having equal subdivisions at lower metrical levels in many cases. In connection with the fact that the measure cycles might extend over a long period of time, these shorter cycles contribute an important aspect to forming what can be perceived as beats. For the evaluations on Carnatic music in this section, we will refer to the subdivision meter and the cycle length as given in \tabref{tab:cm:talastruct}. Since there is no well-defined subdivision meter in Hindustani music, we will refer to only the cycle length in number of \glspl{matra} from \tabref{tab:hm:taalstruct}.

For \acrshort{KLA} and \acrshort{SRI} algorithms we report the accuracy of estimating the annotated cycle length at the \gls{CML}. We also report the \gls{AML} accuracy considering cycle length estimates by the algorithms to be correct that are related to the annotated cycle length by a factor of 2 or 1/2, which is referred to as doubling or halving, respectively. For cycle lengths which are odd we only consider doubling of cycle length estimates in \gls{AML}. Halving and doubling of cycle lengths can be interpreted as estimating sub-cycles and supra-cycles related to the annotated cycle length by a multiple, and can provide insights on tempo estimation errors committed by the algorithms. Though the \gls{tala} cycle is an important part of rhythmic organization, it is not necessary that all phrase changes occur on the \gls{sama}. In \gls{adi} \gls{tala} for example, most of the phrase changes occur at the end of the 8 beat cycle, there are compositions where some phrase changes and strong accents occur at the end of half-cycle or the phrase might span over two cycles (16 beats). Hence, in this case a cycle length of 4, 8, or 16 would be acceptable, depending on the composition. This needs to be considered when we evaluate the performance of algorithms. 

\subsubsection{Self-contained approaches}
We differentiate between self-contained and comparative approaches, and the self-contained approaches are divided into two types of methods. The first type attempts to estimate the meter or the time signature based on repetitions observed in the signals, while the second type aims at tracking the pulsations related to those repetitions. We start our evaluations with methods that belong to the first type (\acrshort{GUL}, \acrshort{PIK}), and evaluate then the tracking methods (\acrshort{KLA}, \acrshort{SRI}). 
\begin{table}
\centering
\begin{tabular}{@{}lr@{}}\toprule
Dataset & Accuracy (\%)  					    \tabularnewline \midrule
Carnatic (without \gls{khanda chapu}) & 75.27  \tabularnewline
Hindustani (without \gls{jhaptal}) & 49.30  	\tabularnewline \bottomrule
\end{tabular}
\caption{Performance of meter estimation using \protect\acrshort{GUL} algorithm.}\label{tab:jnmreval:cycleGUL} 
\end{table} 

\tabref{tab:jnmreval:cycleGUL} shows the accuracies for the two datasets, using the types of rhythms that can be processed by the algorithm. The performance on Carnatic music is better than the performance on Hindustani music. A detailed analysis revealed that the performance on \gls{rupaka} \gls{tala} is only 65.08\%, which leads to considerable decrease in the performance on Carnatic music. This poorer performance can be attributed to the ambiguity between duple and triple meter that is an intrinsic property of this \gls{tala} (see \secref{sec:probdef:challenges}). Furthermore, the performance on Hindustani music was found to be poor on \gls{rupak} and \gls{ektal} while the performance on just \gls{teental} is 80.64\%. This can be attributed to the fact that there are very long cycles in Hindustani music in \gls{vilambit} \gls{lay}, where the long subdivision time-spans restrains the algorithm from a correct estimation. In most of such cases in \gls{ektal} and \gls{rupak}, the estimated meter is a duple meter, which might be related to the further division of the \glspl{matra} using filler strokes. 
\begin{table}
\centering
\begin{tabular}{@{}lrrr@{}}\toprule
Dataset & Method-A & Method-B & Combined\tabularnewline \midrule
Carnatic & 52.53 & 49.30 & 64.06	\tabularnewline
Hindustani & 35.67 & 53.50 & 57.96	\tabularnewline \bottomrule
\end{tabular}
\caption[Performance of cycle length estimation using \protect\acrshort{PIK} algorithm]{Performance of cycle length estimation using \protect\acrshort{PIK} algorithm. The Method-A and Method-B refer to the two methods suggested by \protect\citeA{pikrakis:04:meter}. All values are in percentage.}\label{tab:jnmreval:cyclePIK} 
\end{table}

Pikrakis algorithm (\acrshort{PIK}) looks for measure lengths between 2 and 12. We report the accuracy accepting an answer if it is correct at one of the metrical levels. For example, for \gls{adi} \gls{tala} and \gls{teental}, 4/4, 8/4, 4/8, 8/8 are all evaluated to be correct estimates, because 4 is the subdivision meter, and 8 is the length of the \gls{avartana} (cycle length). Further, the algorithm outputs an estimation for every 5 second frame of audio, and therefore time signature of a song is obtained by using a majority vote for a whole song. The performance is reported as the accuracy of estimation (\% correctly estimated) for both the diagonal processing methods (Method-A and Method-B) in \tabref{tab:jnmreval:cyclePIK}. As suggested by \citeauthor{pikrakis:04:meter}, we also use both methods to combine the decision and it improves the performance, as can be seen from the table. The performance on Carnatic music is better than that on Hindustani music. Though the performance on Hindustani dataset is poor, further analysis shows that for \gls{teental}, the accuracy is 74.19\%. \acrshort{PIK} algorithm performs better in the cases where the meter is a simple duple or triple, while the performance is worse with other meters. For example, \gls{mishra chapu} (length 7) has an additive meter and the cycle can be visualized to be a combination of 3/4 and 4/4. On that class the \acrshort{PIK} algorithm estimates most of \gls{mishra chapu} pieces to have either a 3/4 meter or a 4/4 meter. 

To evaluate the tracking methods, we can compare the pulsations estimated by the algorithms with the ground truth annotations at all three metrical levels to determine if the large possible tempo ranges cause the beat to be tracked at different levels of the meter. From the estimates obtained from \acrshort{KLA} for downbeats, beats and subdivision pulses on a specific piece, we define the following time-spans: let $T_c$ denote the median cycle duration (inter-downbeat interval), $T_b$ the median beat duration, and $T_a$ the median subdivision duration. We use a different terminology for these as compared to $\isi$, $\ibi$, and $\iai$ defined in \secref{sec:probdef:thesismeter} to highlight the difference that these approaches evaluated here were not specifically designed for Indian art music. We then compute the cycle length estimates as, 
\begin{eqnarray}
L_{cb} = \left\lfloor \frac{T_c}{T_b} \right\rceil \nonumber \;\;
L_{ca} = \left\lfloor \frac{T_c}{T_a} \right\rceil \nonumber \;\;
L_{ba} = \left\lfloor \frac{T_b}{T_a} \right\rceil \nonumber
\end{eqnarray}
where $\lfloor.\rceil$ indicates rounding to the nearest integer. We examine which of the three estimates more closely represents the cycle length. We report both the \gls{CML} and \gls{AML} accuracy of cycle length recognition. \tabref{tab:jnmreval:cycleKLA} shows the recognition accuracy (in percentage) of \acrshort{KLA} algorithm separately for $L_{cb}$, $L_{ca}$, or $L_{ba}$ as the cycle length estimates. 
%
\begin{table}
\centering
\begin{tabular}{@{}lcccccccc@{}} \toprule
 & & \multicolumn{3}{c}{\acrshort{CML} (\%)} & & \multicolumn{3}{c}{\acrshort{AML} (\%)}\tabularnewline 
Dataset & & $L_{cb}$ & $L_{ca}$ & $L_{ba}$ & & $L_{cb}$ & $L_{ca}$ & $L_{ba}$ \tabularnewline \midrule
Carnatic & & 11.06 & 8.76 & 4.15 & & 34.10 & 45.16 & 25.81 \tabularnewline \addlinespace[3pt]
Hindustani & & 0.00 & 25.40 & - & & 45.22 & 46.50 & - \tabularnewline \bottomrule
\end{tabular}
\caption[Accuracy of cycle length recognition using \acrshort{KLA} algorithm]{Accuracy of cycle length recognition using \acrshort{KLA} algorithm. Subdivision meter ($L_{ba}$) in Hindustani music is not well-defined and hence omitted.}\label{tab:jnmreval:cycleKLA}
\end{table}
%
\begin{table}
\centering
\begin{tabular}{@{}lcccccccc@{}} \toprule
 & & \multicolumn{3}{c}{\acrshort{CML} (\%)} & & \multicolumn{3}{c}{\acrshort{AML} (\%)}\tabularnewline
Dataset & & $L_{cb}$ & $L_{ca}$ & $L_{ba}$ & & $L_{cb}$ & $L_{ca}$ & $L_{ba}$ \tabularnewline \midrule
Carnatic & & 3.69 & 0.46 & 6.45 & & 40.55 & 50.69 & 14.28 \tabularnewline \addlinespace[3pt]
Hindustani & & 14.64 & 9.55 & - & & 43.95 & 55.41 & - \tabularnewline \bottomrule
\end{tabular}
\caption[Accuracy of cycle length recognition using \acrshort{SRI} algorithm]{Accuracy of cycle length recognition using \acrshort{SRI} algorithm. Subdivision meter ($L_{ba}$) in Hindustani music is not well-defined and hence omitted.}
\label{tab:jnmreval:cycleSRI}
\end{table}

We see in \tabref{tab:jnmreval:cycleKLA} that there is a large difference between \gls{CML} and \gls{AML} performance, which indicates that in many cases tracked level is related to the annotated level by a factor 2 or 1/2. We also see that for Hindustani music, the cycle length is best estimated using $L_{ca}$, with the \gls{CML} accuracy being very low or zero when we use the other cycle length estimates instead. As discussed earlier, in Hindustani music, the cycle length is defined as the number of \glspl{matra} in the cycle. However, in the case of \gls{vilambit} pieces, the \glspl{matra} are longer than the range of the tatum pulse time-span estimated by the algorithm and hence the performance is poor. Interestingly, we see a good performance when evaluated with $L_{cb}$ only with \gls{teental}, which resembles the Eurogenetic 4/4 meter, with an \gls{AML} accuracy of 88.71\% in spite of the \gls{CML} accuracy being zero. In fact, it is seen that $L_{cb}$ is always four in the case of a correct estimation (\gls{AML}), which is the estimate of the number of \glspl{vibhaag} in the \gls{taal}. Further, it follows from \citeA[Figure 8]{klapuri:06:meter} that relation between neighboring levels in \acrshort{KLA} cannot be larger than 9, which implies longer cycle length estimates (as needed by e.g. \gls{ektal} or \gls{teental}) could possibly appear only in the $L_{ca}$ length. 

The \gls{CML} accuracy in Carnatic dataset with $L_{cb}$ is better than the other cycle length estimates, showing that \acrshort{KLA} tracked correct tempo in a majority of cases in Carnatic music. However, the performance is poor because the algorithm often under-estimates the cycle length. Further, in \glspl{tala} of Carnatic music that have two \glspl{akshara} in a beat (\gls{khanda chapu} and \gls{mishra chapu}), $L_{ca}$ is a better indicator of the cycle length than $L_{cb}$, since \glspl{akshara} are closer to the estimated subdivision duration. In general, $L_{ba}$ performs poorly compared to $L_{ca}$ or $L_{cb}$, which is not astonishing since the cycle lengths we are looking for are longer than the estimated subdivision meter. Summing up, none of the estimated meter relations can serve as a robust estimate for the \gls{avartana} cycle length. 

\acrshort{SRI} algorithm estimates the cycle length at two metrical levels using the beats tracked by \citeA{ellis:07:beat} beat tracker, one being at the cycle level (bar length in beats), and the second at the beat level (subdivision meter, or \gls{nade}). The algorithm computes a list of possible candidates for the subdivision meter and bar length, ordered by a score. We consider the top candidate in the list and compute the cycle length estimates $L_{cb}$, $L_{ba}$, and the $L_{ca}$, assuming that the beats tracked by Ellis beat tracker correspond the beat duration $T_b$. Similar to \acrshort{KLA} algorithm, we present the \gls{CML} and \gls{AML} accuracy of performance in \tabref{tab:jnmreval:cycleSRI}. 

We see that there is large disparity between the \gls{CML} and \gls{AML} accuracy, which indicates that the beat tracker and the correct beat are related by a factor of $2$ or $1/2$. In general, the algorithm performs poorly, which can be mainly attributed to errors in tempo and beat tracking. The tempo estimation uses a weighting curve that peaks at 90 beats per minute, which is suitable for Carnatic music, but leads to an incorrect estimation of cycle length for Hindustani music. A beat tracking based approach as the \acrshort{SRI} algorithm might in general not be well suited for Hindustani music which often includes long cycles. % that might be more reflected in the structure of melodic phrases than in pulsation and rhythmic aspects. 

The poor performance on Carnatic music can in part be also attributed to variation in percussion accompaniment, which is completely free to improvise within the framework of the \gls{tala}. Further, the algorithm is based on the implicit assumption that beats at the same position in a measure cycle are similar between various recurrences of the cycle. For certain music pieces where there are no inherent rhythmic patterns or the patterns vary unpredictably, the algorithm gives a poorer performance. For Carnatic music, the algorithm specifically estimates the subdivision-meter (\gls{nade}), as the number of \glspl{akshara} per beat. Using $L_{ba}$ as an estimate of the \gls{nade}, we obtain a reasonably good performance comparable to \acrshort{GUL} with an accuracy of 39.63\% and 79.72\% at \gls{CML} and \gls{AML} (of the subdivision meter), respectively. We see that a reasonable performance when demanding an exact numerical result for the meter (\gls{CML}) is only reached for the \gls{nade} estimation in Carnatic music.

We observe that the duration of cycles in seconds is often estimated correctly, but the presence or absence of extra beats causes the estimated length in beats to be wrong. Ellis beat tracker is sensitive to tempo value and cannot handle small tempo changes effectively. This leads to addition of beats into the cycle and the cycle length in many cases were estimated to be one-off from the actual value, though the actual duration of the cycle (in seconds) was estimated correctly. 
%
\subsubsection{Comparative approaches}
The comparative approaches are based on a description of periodicities that can be derived from the signal without the need to perform meter tracking. Performances of the two evaluated methods, \acrshort{OP} and \acrshort{STM}, is the average accuracy in a 1-nearest neighbor classification. It tells us how often a piece found to be most similar to a test piece belongs actually to the same class of rhythm as the test piece. The results of this classification experiment are depicted in \tabref{tab:jnmreval:cycleCOMP}. It is apparent that the comparative approaches lead to a performance significantly better than random, which would be 25\% for our compiled four-class datasets. In fact, accuracies are in the same range as the results of the \acrshort{PIK} algorithm, with \acrshort{PIK} performing better on Carnatic music (64.1\% instead of 42.2\%). This might indicate the potential of combining self-contained and comparative approaches, because none of the approaches evaluated for cycle length recognition provide us with a sufficient performance for a practical application. 
\begin{table}
\centering
\begin{tabular}{@{}lcr@{}} \toprule
Dataset & \acrshort{OP} (\%)  & \acrshort{STM} (\%) \tabularnewline \midrule
Carnatic & 41.0  & 42.2    \tabularnewline 
Hindustani & 47.8  & 51.6   \tabularnewline \bottomrule
\end{tabular}
\caption{Accuracy of cycle length recognition using comparative approaches}\label{tab:jnmreval:cycleCOMP} 
\end{table}