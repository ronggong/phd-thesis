\newcommand\resource[2]{
\noindent #1 \par
\vspace{0.2em}
{\centering	\url{#2} \par}
\vspace{0.5em}
\hrule \par 
\vspace{0.8em} \par}
%

\chapter[List of Publications][List of Publications]{List of Publications}\label{app:mypapers}% by the author

\subsection*{First author full articles in peer-reviewed conferences}
\begin{itemize}[leftmargin=*]
	\item Gong, R., Cuvillier, P., Obin, N., \& Cont, A. (2015, September). Real-time audio-to-score alignment of singing voice based on melody and lyric information. \textit{In Interspeech 2015}; Dresden, Germany. \newline \url{https://hal.archives-ouvertes.fr/hal-01164550}
	\item Gong, R., Yang, Y., \& Serra, X. (2016). Pitch contour segmentation for computer-aided jinju singing training. \textit{13th Sound \& Music Computing Conference}; 2016 Aug 31-Sep 3; Hamburg, Germany. \newline \url{http://mtg.upf.edu/node/3537}
	\item Gong, R., Obin, N., Dzhambazov, G. B., \& Serra, X. (2017). Score-informed syllable segmentation for jingju a cappella singing voice with mel-frequency intensity profiles. \textit{In Proceedings of the 7th International Workshop on Folk Music Analysis}; 2017 Jun 14-16; Málaga, Spain. \newline \url{https://doi.org/10.5281/zenodo.556820}
	\item Gong, R., Pons, J., \& Serra, X. (2017). Audio to score matching by combining phonetic and duration information. \textit{In ISMIR}; 2017 Sep; Suzhou, China. \newline \url{https://arxiv.org/abs/1707.03547}
	\item Gong, R., Repetto, R. C., \& Serra, X. (2017, October). Creating an A Cappella Singing Audio Dataset for Automatic Jingju Singing Evaluation Research. \textit{In Proceedings of the 4th International Workshop on Digital Libraries for Musicology}; 2017 Sep; Shanghai, China. \newline \url{https://doi.org/10.1145/3144749.3144757}
	\item Gong, R., \& Serra, X. (2017). Identification of potential Music Information Retrieval technologies for computer-aided jingju singing training. \textit{In Chinese traditional music technology session - China conference on sound and music technology}; 2017 Sep; Suzhou, China. \newline \url{https://arxiv.org/abs/1711.07551}
	\item Gong, R., \& Serra, X. (2018). Singing voice phoneme segmentation by hierarchically inferring syllable and phoneme onset positions. \textit{In Interspeech 2018}; Hyderabad, India. \newline \url{https://arxiv.org/abs/1806.01665}
\end{itemize} 

\subsection*{Second author full articles in peer-reviewed conferences}
\begin{itemize}[leftmargin=*]
	\item Caro Repetto, R., Gong, R., Kroher, N., \& Serra, X. (2015). Comparision of the singing style of two jingju schools. \textit{16th International Society for Music Information Retrieval Conference}; 2015 Oct 26-30; Málaga, Spain. \newline \url{http://mtg.upf.edu/node/3317}
	\item Fonseca, E., Gong, R., Bogdanov, D., Slizovskaia, O., Gómez Gutiérrez, E., \& Serra, X. (2017). Acoustic scene classification by ensembling gradient boosting machine and convolutional neural networks. \textit{Detection and Classification of Acoustic Scenes and Events 2017 Workshop (DCASE2017)}; 2017 Nov 16; Munich, Germany. \newline \url{http://hdl.handle.net/10230/33454}
	\item Pons, J., Gong, R., \& Serra, X. (2017). Score-informed syllable segmentation for a cappella singing voice with convolutional neural networks. \textit{In ISMIR 2017}; 2017 Sep; Suzhou, China. \newline \url{https://arxiv.org/abs/1707.03544}
	\item Fonseca, E., Gong, R., \& Serra, X. (2018). A Simple Fusion of Deep and Shallow Learning for Acoustic Scene Classification. \textit{15th Sound \& Music Computing Conference}; 2018 July; Limassol, Cyprus. \newline \url{https://arxiv.org/abs/1806.07506}
\end{itemize}

\subsection*{Third author full articles in peer-reviewed conferences}
\begin{itemize}[leftmargin=*]
	\item Pons, J., Slizovskaia, O., Gong, R., Gómez, E., \& Serra, X. (2017, August). Timbre analysis of music audio signals with convolutional neural networks. \textit{In EUSIPCO 2017}; Kos, Greece. \newline \url{https://arxiv.org/abs/1703.06697}
\end{itemize}


\chapter{Resources}\label{app:resources}
% <<Whatever could not make it to the main text>>
This appendix is a compendium of links to resources and additional material related to the work presented in the thesis. An up-to-date set of links is also listed and maintained on the companion webpage \url{http://compmusic.upf.edu/phd-thesis-rgong}.

Some of the results not reported in the dissertation are presented on the companion webpage. The companion webpage will also be updated with any additional resources and material that will be built in the future. 

\section*{Corpora and datasets}
Access to the corpora and datasets will be through the Zenodo.org and MusicBrainz.org
% \begin{description}
% \item[Research corpora] - \url{http://compmusic.upf.edu/corpora}
% \item[Test datasets] - \url{http://compmusic.upf.edu/datasets}
% \end{description}

\subsection*{Research corpus}

\resource{Jingju a cappella singing voice dataset part 1}{https://doi.org/10.5281/zenodo.780559}
% 
\resource{Jingju a cappella singing voice dataset part 2}{https://doi.org/10.5281/zenodo.842229}

\resource{Jingju a cappella singing voice dataset part 2}{https://doi.org/10.5281/zenodo.1244732}

\resource{Jingju a cappella singing voice dataset metadata on MusicBrainz}{https://musicbrainz.org/search?query=MTG-UPF&type=release&method=indexed}

\subsection*{Test dataset}
% 
\resource{Automatic syllable and phoneme segmentation test dataset part 1 -- ASPS\textsubscript{1}}{https://doi.org/10.5281/zenodo.1185123}

\resource{Automatic syllable and phoneme segmentation test dataset part 2 -- ASPS\textsubscript{2}}{https://doi.org/10.5281/zenodo.1341070}

\resource{Pronunciation and overall quality similarity measures test dataset -- POQSM}{https://doi.org/10.5281/zenodo.1287251}

\section*{Code}
The links to code related to the thesis are listed. Up-to-date links to code (including future releases) will be available on: \url{https://github.com/ronggong}
\begin{description}[style=nextline,font=\normalfont]
\item[Automatic syllable and phoneme segmentation baseline code] \url{https://github.com/ronggong/interspeech2018_submission01}
\item[Automatic syllable and phoneme segmentation onset detection function improvement code] \url{https://github.com/ronggong/musical-onset-efficient}
\item[Mispronunciation detection code] \url{https://github.com/ronggong/mispronunciation-detection}
\item[Pronunciation and overall quality similarity measures code] \url{https://github.com/ronggong/DLfM2018}
\end{description}
% Discuss specific datasets and code, with links, explain the dataset organization, access. 
%one idea: identify recorded cycles which are very close to the depicted patterns, and provide them as acoustic documentation on a website and multimedia appendix to the thesis