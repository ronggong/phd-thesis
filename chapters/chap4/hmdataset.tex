\acrfull{HMDf}\footnote{\url{http://compmusic.upf.edu/hindustani-rhythm-dataset}}\index{Hindustani music} is a rhythm annotated test corpus for automatic rhythm analysis tasks in Hindustani Music~\cite{ajay:16:spmodel}. The collection consists of audio excerpts from the CompMusic Hindustani research corpus, manually annotated time aligned markers indicating the progression through the \gls{taal} cycle, and the associated \gls{taal} related metadata. The dataset has pieces from four popular \glspl{taal} of Hindustani music (\tabref{tab:dataset:hmdf}), which encompasses a majority of Hindustani \gls{khayal} music. 

The audio recordings are chosen from the CompMusic Hindustani music research corpus. The pieces include a mix of vocal and instrumental recordings, new and old recordings, and span three \gls{lay} classes. For each taal, there are pieces in \gls{dhrut} (fast), \gls{madhyam} (medium) and \gls{vilambit} \gls{lay}. All pieces have \gls{tabla} as the percussion accompaniment. All the audio recordings in the dataset are 2 minute excerpts of full length pieces. Each piece is uniquely identified using the \gls{MBID} of the recording. The pieces are stereo, 160 kbps, mp3 files sampled at 44.1 kHz. The audio is also available as downmixed mono WAV files for experiments. 
%% HMDf table of dataset
\begin{table}[t]
\begin{center}
\begin{tabular}{@{}lrcrr@{}}
\toprule 
\Gls{taal} & \# Pieces & Total Duration & \# Ann. & \# Sam\tabularnewline
 &  & hours (min) &  & \tabularnewline
\midrule 
\Gls{teental} & 54 & 1.80 (108) & 17142 & 1081\tabularnewline
\Gls{ektal} & 58 & 1.93 (116) & 12999 & 1087\tabularnewline
\Gls{jhaptal} & 19 & 0.63 (38) & 3029 & 302\tabularnewline
\Gls{rupak} & 20 & 0.67 (40) & 2841 & 406\tabularnewline
\midrule 
Total & 151 & 5.03 (302) & 36011 & 2876\tabularnewline
\bottomrule 
\end{tabular}
\end{center}
\protect\caption[\acrshort{HMDf} dataset description]{\acrshort{HMDf} dataset showing the total duration and number of annotations. \#Sam shows the number of \gls{sam} annotations and \#Ann. shows the number of \gls{matra} annotations (including \glspl{sam}).}
\label{tab:dataset:hmdf}
\end{table}
%% HMDf table of datastats
\begin{table}[t]
\begin{center}
\begin{tabular}{@{}lccc@{}}
\toprule 
\Gls{taal} & $\overline{\isi} \pm \sigma_s$ & $\overline{\ibi} \pm \sigma_b$  & $[{\isi}_{,\min} \, , \,  {\isi}_{,\max} ]$ \tabularnewline
\midrule 
\Gls{teental} & 10.36 $\pm$ 9.875 & 0.65 $\pm$ 0.617 & {[}2.32, 44.14{]}\tabularnewline
\Gls{ektal} & 30.20 $\pm$ 26.258 & 2.52 $\pm$ 2.188 & {[}2.23, 69.73{]}\tabularnewline
\Gls{jhaptal} & 8.51 $\pm$ 3.149 & 0.85 $\pm$ 0.315 & {[}4.06, 16.23{]}\tabularnewline
\Gls{rupak} & 7.11 $\pm$ 3.360 & 1.02 $\pm$ 0.480 & {[}2.82, 16.09{]}\tabularnewline
\bottomrule 
\end{tabular}
\end{center}
\protect\caption[\Gls{taal} cycle length indicators for \acrshort{HMDf} dataset]{\Gls{taal} cycle length indicators for \acrshort{HMDf} dataset. $\overline{\isi}$ and $\sigma_s$ indicate the mean and standard deviation of the median inter-\gls{sam} interval of the pieces, respectively. $\overline{\ibi}$ and $\sigma_b$ indicate the mean and standard deviation of the median inter-\gls{matra} interval of the pieces, respectively. $[{\isi}_{,\min} \, , \,  {\isi}_{,\max}]$ indicate the minimum and maximum value of $\isi$ and hence the range of $\isi$ in the dataset. All values in the table are in seconds.}
\label{tab:datastat:hmdf}
\end{table}
%% HMDl table of dataset
\begin{table}[t]
\begin{center}
\begin{tabular}{@{}lrcrr@{}}
\toprule 
\Gls{taal} & \# Pieces & Total Duration & \# Ann. & \# Sam\tabularnewline
 &  & hours (min) &  & \tabularnewline
\midrule 
\Gls{teental} & 13 & 0.43 (26) & 1020 & 65\tabularnewline
\Gls{ektal} & 32 & 1.07 (64) & 967 & 79\tabularnewline
\Gls{jhaptal} & 6 & 0.2 (12) & 592 & 59\tabularnewline
\Gls{rupak} & 8 & 0.27 (16) & 701 & 101\tabularnewline
\midrule 
Total & 59 & 1.97 (118) & 3280 & 304\tabularnewline
\bottomrule 
\end{tabular}
\end{center}
\protect\caption[\acrshort{HMDl} dataset description]{\acrshort{HMDl} dataset showing the total duration and number of annotations. \#Sam shows the number of \gls{sam} annotations and \#Ann. shows the number of \gls{matra} annotations (including \glspl{sam}).}
\label{tab:dataset:hmdl}
\end{table}
%% HMDl table of datastats
\begin{table}[t]
\begin{center}
\begin{tabular}{@{}lccc@{}}
\toprule 
\Gls{taal} & $\overline{\isi} \pm \sigma_s$ & $\overline{\ibi} \pm \sigma_b$  & $[{\isi}_{,\min} \, , \,  {\isi}_{,\max} ]$ \tabularnewline
\midrule 
\Gls{teental} & 26.16 $\pm$ 7.963  & 1.63 $\pm$ 0.498 & {[}18.57, 44.14{]}\tabularnewline
\Gls{ektal} & 52.16 $\pm$ 12.531 & 4.35 $\pm$ 1.044 & {[}14.43, 69.73{]}\tabularnewline
\Gls{jhaptal} & 12.30 $\pm$ 1.935 & 1.23 $\pm$ 0.194 & {[}10.20, 16.23{]}\tabularnewline
\Gls{rupak} & 10.28 $\pm$ 3.050 & 1.47 $\pm$ 0.436 & {[}6.95, 16.09{]}\tabularnewline
\bottomrule 
\end{tabular}
\end{center}
\protect\caption[\Gls{taal} cycle length indicators for \acrshort{HMDl} dataset]{\Gls{taal} cycle length indicators for \acrshort{HMDl} dataset. $\overline{\isi}$ and $\sigma_s$ indicate the mean and standard deviation of the median inter-\gls{sam} interval of the pieces, respectively. $\overline{\ibi}$ and $\sigma_b$ indicate the mean and standard deviation of the median inter-\gls{matra} interval of the pieces, respectively. $[{\isi}_{,\min} \, , \,  {\isi}_{,\max}]$ indicate the minimum and maximum value of $\isi$ and hence the range of $\isi$ in the dataset. All values in the table are in seconds.}
\label{tab:datastat:hmdl}
\end{table}
%% HMDs table of dataset
\begin{table}[t]
\begin{center}
\begin{tabular}{@{}lrcrr@{}}
\toprule 
\Gls{taal} & \# Pieces & Total Duration & \# Ann. & \# Sam\tabularnewline
 &  & hours (min) &  & \tabularnewline
\midrule 
\Gls{teental} & 41 & 1.37 (82) & 16122 & 1016\tabularnewline
\Gls{ektal} & 26 & 0.87 (52) & 12032 & 1008\tabularnewline
\Gls{jhaptal} & 13 & 0.43 (26) & 2437 & 243\tabularnewline
\Gls{rupak} & 12 & 0.40 (24) & 2140 & 305\tabularnewline
\midrule 
Total & 92 & 3.07 (184) & 32731 & 2572\tabularnewline
\bottomrule 
\end{tabular}
\end{center}
\protect\caption[\acrshort{HMDs} dataset description]{\acrshort{HMDs} dataset showing the total duration and number of annotations. \#Sam shows the number of \gls{sam} annotations and \#Ann. shows the number of \gls{matra} annotations (including \glspl{sam}).}
\label{tab:dataset:hmds}
\end{table}
%% HMDs table of datastats
\begin{table}[t]
\begin{center}
\begin{tabular}{@{}lccc@{}}
\toprule 
\Gls{taal} & $\overline{\isi} \pm \sigma_s$ & $\overline{\ibi} \pm \sigma_b$  & $[{\isi}_{,\min} \, , \,  {\isi}_{,\max} ]$ \tabularnewline
\midrule 
\Gls{teental} & 5.35 $\pm$ 1.823 & 0.33 $\pm$ 0.114 & {[}2.32, 9.89{]}\tabularnewline
\Gls{ektal} & 3.17 $\pm$ 0.471 & 0.26 $\pm$ 0.039 & {[}2.23, 4.11{]}\tabularnewline
\Gls{jhaptal} & 6.77 $\pm$ 1.688 & 0.68 $\pm$ 0.169 & {[}4.06, 9.97{]}\tabularnewline
\Gls{rupak} & 5.00 $\pm$ 1.191 & 0.71 $\pm$ 0.170 & {[}2.82, 6.68{]}\tabularnewline
\bottomrule 
\end{tabular}
\end{center}
\protect\caption[\Gls{taal} cycle length indicators for \acrshort{HMDs} dataset]{\Gls{taal} cycle length indicators for \acrshort{HMDs} dataset. $\overline{\isi}$ and $\sigma_s$ indicate the mean and standard deviation of the median inter-\gls{sam} interval of the pieces, respectively. $\overline{\ibi}$ and $\sigma_b$ indicate the mean and standard deviation of the median inter-\gls{matra} interval of the pieces, respectively. $[{\isi}_{,\min} \, , \,  {\isi}_{,\max}]$ indicate the minimum and maximum value of $\isi$ and hence the range of $\isi$ in the dataset. All values in the table are in seconds.}
\label{tab:datastat:hmds}
\end{table}
% Dataset stats: HMDl-ISI
\begin{figure}[t]
%\captionsetup[subfigure]{labelformat=empty}
\centering
\subfloat[\Gls{teental}]{\label{fig:dstats:HMDl:ISI:teen}\includegraphics[scale=0.9]{dstats/HMDl-teen-ISI.pdf}} \hspace{0.5cm} 
\subfloat[\Gls{ektal}]{\label{fig:dstats:HMDl:ISI:ek}\includegraphics[scale=0.9]{dstats/HMDl-ek-ISI.pdf}} \\ 
\subfloat[\Gls{jhaptal}]{\label{fig:dstats:HMDl:ISI:jhap}\includegraphics[scale=0.9]{dstats/HMDl-jhap-ISI.pdf}} \hspace{0.5cm} 
\subfloat[\Gls{rupak}]{\label{fig:dstats:HMDl:ISI:rupak}\includegraphics[scale=0.9]{dstats/HMDl-rupak-ISI.pdf}} \\ 
\caption[Histogram of $\protect\isi$ in the \acrshort{HMDl} dataset]{A histogram of the inter-\gls{sam} interval $\protect\isi$ in the \acrshort{HMDl} dataset for each \gls{taal}. The ordinate is the fraction of the total count corresponding to the $\ibi$ value shown in abscissa. The median $\isi$ for each \gls{taal} is shown as a red dotted line.}\label{fig:dstats:HMDl:ISI}
\end{figure}
%
% Dataset stats: HMDl-IAI
\begin{figure}[t]
%\captionsetup[subfigure]{labelformat=empty}
\centering
\subfloat[\Gls{teental}]{\label{fig:dstats:HMDl:IAI:teen}\includegraphics[scale=0.9]{dstats/HMDl-teen-IAI.pdf}} \hspace{0.5cm} 
\subfloat[\Gls{ektal}]{\label{fig:dstats:HMDl:IAI:ek}\includegraphics[scale=0.9]{dstats/HMDl-ek-IAI.pdf}} \\ 
\subfloat[\Gls{jhaptal}]{\label{fig:dstats:HMDl:IAI:jhap}\includegraphics[scale=0.9]{dstats/HMDl-jhap-IAI.pdf}} \hspace{0.5cm} 
\subfloat[\Gls{rupak}]{\label{fig:dstats:HMDl:IAI:rupak}\includegraphics[scale=0.9]{dstats/HMDl-rupak-IAI.pdf}} \\ 
\caption[Histogram of $\protect\ibi$ in the \acrshort{HMDl} dataset]{A histogram of the inter-\gls{matra} interval $\protect\ibi$ in the \acrshort{HMDl} dataset for each \gls{taal}. The ordinate is the fraction of the total count corresponding to the $\ibi$ value shown in abscissa. The median $\ibi$ for each \gls{taal} is shown as a red dotted line.}\label{fig:dstats:HMDl:IAI}
\end{figure}
%
% Dataset stats: HMDs-ISI
\begin{figure}[t]
%\captionsetup[subfigure]{labelformat=empty}
\centering
\subfloat[\Gls{teental}]{\label{fig:dstats:HMDs:ISI:teen}\includegraphics[scale=0.9]{dstats/HMDs-teen-ISI.pdf}} \hspace{0.5cm} 
\subfloat[\Gls{ektal}]{\label{fig:dstats:HMDs:ISI:ek}\includegraphics[scale=0.9]{dstats/HMDs-ek-ISI.pdf}} \\ 
\subfloat[\Gls{jhaptal}]{\label{fig:dstats:HMDs:ISI:jhap}\includegraphics[scale=0.9]{dstats/HMDs-jhap-ISI.pdf}} \hspace{0.5cm} 
\subfloat[\Gls{rupak}]{\label{fig:dstats:HMDs:ISI:rupak}\includegraphics[scale=0.9]{dstats/HMDs-rupak-ISI.pdf}} \\ 
\caption[Histogram of $\protect\isi$ in the \acrshort{HMDs} dataset]{A histogram of the inter-\gls{sam} interval $\protect\isi$ in the \acrshort{HMDs} dataset for each \gls{taal}. The ordinate is the fraction of the total count corresponding to the $\ibi$ value shown in abscissa. The median $\isi$ for each \gls{taal} is shown as a red dotted line.}\label{fig:dstats:HMDs:ISI}
\end{figure}
%
% Dataset stats: HMDs-IAI
\begin{figure}[t]
%\captionsetup[subfigure]{labelformat=empty}
\centering
\subfloat[\Gls{teental}]{\label{fig:dstats:HMDs:IAI:teen}\includegraphics[scale=0.9]{dstats/HMDs-teen-IAI.pdf}} \hspace{0.5cm} 
\subfloat[\Gls{ektal}]{\label{fig:dstats:HMDs:IAI:ek}\includegraphics[scale=0.9]{dstats/HMDs-ek-IAI.pdf}} \\ 
\subfloat[\Gls{jhaptal}]{\label{fig:dstats:HMDs:IAI:jhap}\includegraphics[scale=0.9]{dstats/HMDs-jhap-IAI.pdf}} \hspace{0.5cm} 
\subfloat[\Gls{rupak}]{\label{fig:dstats:HMDs:IAI:rupak}\includegraphics[scale=0.9]{dstats/HMDs-rupak-IAI.pdf}} \\ 
\caption[Histogram of $\protect\ibi$ in the \acrshort{HMDs} dataset]{A histogram of the inter-\gls{matra} interval $\protect\ibi$ in the \acrshort{HMDs} dataset for each \gls{taal}. The ordinate is the fraction of the total count corresponding to the $\ibi$ value shown in abscissa. The median $\ibi$ for each \gls{taal} is shown as a red dotted line.}\label{fig:dstats:HMDs:IAI}
\end{figure}
%
% Dataset stats: HMDl-ISInorm
\begin{figure}[t]
%\captionsetup[subfigure]{labelformat=empty}
\centering
\subfloat[\Gls{teental}]{\label{fig:dstats:HMDl:ISInorm:teen}\includegraphics[scale=0.9]{dstats/HMDl-teen-ISInorm.pdf}} \hspace{0.5cm} 
\subfloat[\Gls{ektal}]{\label{fig:dstats:HMDl:ISInorm:ek}\includegraphics[scale=0.9]{dstats/HMDl-ek-ISInorm.pdf}} \\ 
\subfloat[\Gls{jhaptal}]{\label{fig:dstats:HMDl:ISInorm:jhap}\includegraphics[scale=0.9]{dstats/HMDl-jhap-ISInorm.pdf}} \hspace{0.5cm} 
\subfloat[\Gls{rupak}]{\label{fig:dstats:HMDl:ISInorm:rupak}\includegraphics[scale=0.9]{dstats/HMDl-rupak-ISInorm.pdf}} \\ 
\caption[Histogram of median normalized $\protect\isi$ in the \acrshort{HMDl} dataset]{A histogram of the median normalized inter-\gls{sam} interval $\protect\isi$ in the \acrshort{HMDl} dataset for each \gls{taal}. The ordinate is the fraction of the total count corresponding to the normalized $\isi$ value shown in abscissa.}\label{fig:dstats:HMDl:ISInorm}
\end{figure}
%
% Dataset stats: HMDl-IAInorm
\begin{figure}[t]
%\captionsetup[subfigure]{labelformat=empty}
\centering
\subfloat[\Gls{teental}]{\label{fig:dstats:HMDl:IAInorm:teen}\includegraphics[scale=0.9]{dstats/HMDl-teen-IAInorm.pdf}} \hspace{0.5cm} 
\subfloat[\Gls{ektal}]{\label{fig:dstats:HMDl:IAInorm:ek}\includegraphics[scale=0.9]{dstats/HMDl-ek-IAInorm.pdf}} \\ 
\subfloat[\Gls{jhaptal}]{\label{fig:dstats:HMDl:IAInorm:jhap}\includegraphics[scale=0.9]{dstats/HMDl-jhap-IAInorm.pdf}} \hspace{0.4cm} 
\subfloat[\Gls{rupak}]{\label{fig:dstats:HMDl:IAInorm:rupak}\includegraphics[scale=0.9]{dstats/HMDl-rupak-IAInorm.pdf}} \\ 
\caption[Histogram of median normalized $\protect\ibi$ in the \acrshort{HMDl} dataset]{A histogram of the median normalized inter-\gls{matra} interval $\protect\ibi$ in the \acrshort{HMDl} dataset for each \gls{taal}. The ordinate is the fraction of the total count corresponding to the normalized $\ibi$ value shown in abscissa.}\label{fig:dstats:HMDl:IAInorm}
\centering
\end{figure}
%

There are several annotations that accompany each audio file in the dataset. The primary annotations are audio synchronized time-stamps indicating the different metrical positions in the \gls{taal} cycle. The \gls{sam} and \glspl{matra} of the cycle are annotated. The annotations were created using Sonic Visualizer by tapping to music and manually correcting the taps. Each annotation has a time-stamp and an associated numeric label that indicates the \gls{matra} position in the \gls{taal} cycle illustrated in \figref{fig:taal:hindustani}. The \glspl{sam} are indicated using the numeral 1. The time varying tempo of the piece can be obtained from the \gls{matra} and \gls{sam} annotations. 

For each excerpt, the \gls{taal} and the \gls{lay} of the piece are recorded. Each excerpt can be uniquely identified and located with the \gls{MBID} of the recording, and the relative start and end times of the excerpt within the whole recording. The artist, release, the lead instrument, and the \gls{raag} of the piece are additional editorial metadata obtained from the release. There are optional comments on audio quality and annotation specifics. The annotations and the associated metadata have been verified for correctness and completeness by a professional Hindustani musician and musicologist. 

The \acrshort{HMDf} dataset is described in \tabref{tab:dataset:hmdf}, showing the four \glspl{taal} and the number of pieces for each \gls{taal}, totaling to 151 pieces. The total duration of audio in the dataset is about 5 hours, with 36011 time-aligned \gls{matra} annotations of which 2876 are \gls{sam} annotations. \tabref{tab:datastat:hmdf} shows a basic statistical analysis of the \gls{taal} cycle length indicators in the dataset to understand the tempo characteristics and the range of the metrical cycle lengths in the dataset. The large range of tempi seen in Hindustani music is reflected in the dataset, with the values of median inter-\gls{sam} interval $\overline{\isi}$, \gls{ektal} cycle lengths ranging from 2.2 seconds to 69.7 seconds, which is about 5 tempo octaves. This also shows that the \gls{matra} period can vary from less than 150 ms to over 6 seconds. This huge range of cycle lengths and \gls{matra} periods is a significant challenge in automatic meter analysis of Hindustani music. Across different \glspl{taal}, we see that \gls{teental} and \gls{ektal} have the largest range of $\overline{\isi}$, since they are performed in all the \gls{lay} classes, \gls{vilambit} to \gls{dhrut}. \Gls{jhaptal} and \gls{rupak} have smaller $\overline{\isi}$ ranges. 

The dataset consists of excerpts with a wide tempo range from 10 \mpm\ (\glspl{matra} per minute) to 370 \mpm. As discussed in \chapref{chap:bkgnd}, Hindustani music divides tempo into three main tempo classes (\gls{lay}). Since no exact tempo ranges are defined for these classes, we determined suitable values, measured in \glspl{matra} per minute (\mpm), in correspondence with a professional Hindustani musician as 10-60 \mpm, 60-150 \mpm, and $>$150 \mpm\ for the slow (\gls{vilambit}), medium (\gls{madhyam}), and fast (\gls{dhrut}) tempi, respectively. 

The \gls{lay} of a piece has a significant effect on meter tracking and rhythm analysis due to this wide range of possible tempo. To study any effects of the tempo class, the full \acrshort{HMDf} dataset is divided into two other subsets - the long cycle duration subset called the \acrshort{HMDl} dataset (shown in \tabref{tab:dataset:hmdl}) consisting of \gls{vilambit} pieces with a median tempo between 10-60 \mpm, and the short cycle duration subset \acrshort{HMDs} dataset (shown in \tabref{tab:dataset:hmds}) with \gls{madhyam} \gls{lay} (60-150 \mpm) and the \gls{dhrut} \gls{lay} (150+ \mpm) pieces. 

\acrshort{HMDl} dataset shown in \tabref{tab:dataset:hmdl} consists of 59 pieces in \gls{vilambit} \gls{lay}, with over 3200 \gls{matra} and \gls{sam} annotations. A majority of pieces are in \gls{ektal} and \gls{teental}. Since its very uncommon for a piece to be performed in \gls{vilambit} \gls{lay} \gls{jhaptal} and \gls{rupak}, there are only 6 and 8 pieces for those \glspl{taal}, respectively. As described with \acrshort{HMDf}, a basic statistical analysis of the \gls{taal} cycle length indicators in \tabref{tab:datastat:hmdl} shows that the median inter-\gls{sam} interval and its range for \gls{jhaptal} and \gls{rupak} are less than that for \gls{teental} and \gls{ektal}.

\acrshort{HMDs} dataset in \tabref{tab:dataset:hmds} consists on 92 pieces in \gls{madhyam} and \gls{dhrut} \gls{lay}, with over 3 hours of audio and over 32700 \gls{matra} and \gls{sam} annotations. A basic statistical analysis of the \gls{taal} cycle length indicators in \tabref{tab:datastat:hmds} shows that the pieces of \gls{teental} and \gls{ektal} have higher tempi in the dataset. Comparing the median \gls{matra} period for \gls{ektal} between \tabref{tab:datastat:hmdl} (4.35 second) and \tabref{tab:datastat:hmds} (0.26 second) shows that \gls{ektal} is performed either in \gls{vilambit} or \gls{dhrut} and its rare for a piece to be performed in \gls{madhyam} \gls{lay} \gls{ektal}.

The pieces is Hindustani music have a tempo class indicated but not a specific tempo value, nor are they performed to a metronome. The tempo varies over a piece in time - often the tempo increases with time. Hence, in addition to the median values tabulated in \tabref{tab:datastat:hmdf} we present further analysis of the inter-\gls{sam} interval (\isi) and inter-\gls{matra} interval (\ibi) for each \gls{taal}. For better comparison, we present this analysis for each data subset \acrshort{HMDl} and \acrshort{HMDs} separately. 

A histogram of $\isi$ and $\ibi$ for each \gls{taal} for \acrshort{HMDl} dataset is shown in \figref{fig:dstats:HMDl:ISI} and \figref{fig:dstats:HMDl:IAI}, respectively, and those for \acrshort{HMDs} dataset is shown in \figref{fig:dstats:HMDs:ISI} and \figref{fig:dstats:HMDs:IAI}, respectively. These figures show the distribution of cycle lengths in the dataset over the whole range of $\isi$ for each \gls{taal}, around the median value. The large range of $\isi$ and $\ibi$ values and an irregular distribution spanning the whole range is seen with both datasets, unlike the Carnatic music \acrshort{CMDf} dataset with a smaller tightly defined range of tempo. 

In addition, similar to what was presented for Carnatic music, to illustrate and measure the time varying tempo of music pieces in Hindustani music, we normalize all the $\isi$ and $\ibi$ values in a piece by the median in the piece to obtain median normalized $\isi$ and $\ibi$ values, a histogram of which is shown in \figref{fig:dstats:HMDl:ISInorm} and \figref{fig:dstats:HMDl:IAInorm}, respectively for \acrshort{HMDl} dataset and \figref{fig:dstats:HMDs:ISInorm} and \figref{fig:dstats:HMDs:IAInorm}, respectively for \acrshort{HMDs} dataset. These histograms are centered around 1 and normalized by the median. 

From the figures, it is clear that the tempo is time varying but with less than about 10\% maximum deviation from the median tempo of the piece for all \glspl{taal}. This is in contrast to Carnatic music where the median normalized tempo had a higher maximum deviation ($\sim$ 20 \%). One possible reason for this lower tempo deviation in Hindustani music compared to Carnatic music is because of less rhythmic improvisation, with the tabla acting as an accurate timekeeper. However, this could also be possibly due to the fact that the Hindustani pieces in the dataset are two minute short excerpts, compared to full length Carnatic pieces in the \acrshort{CMDf} dataset, and hence have lower tempo variability. 
%% Dataset stats: HMDl-IAI
\begin{figure}[t]
%\captionsetup[subfigure]{labelformat=empty}
\centering
\subfloat[\Gls{teental}]{\label{fig:dstats:HMDl:IAI:teen}\includegraphics[scale=1]{dstats/HMDl-teen-IAI.pdf}} \hspace{0.5cm} 
\subfloat[\Gls{ektal}]{\label{fig:dstats:HMDl:IAI:ek}\includegraphics[scale=1]{dstats/HMDl-ek-IAI.pdf}} \\ 
\subfloat[\Gls{jhaptal}]{\label{fig:dstats:HMDl:IAI:jhap}\includegraphics[scale=1]{dstats/HMDl-jhap-IAI.pdf}} \hspace{0.5cm} 
\subfloat[\Gls{rupak}]{\label{fig:dstats:HMDl:IAI:rupak}\includegraphics[scale=1]{dstats/HMDl-rupak-IAI.pdf}} \\ 
\caption[HMDl-IAI]{HMDl-IAI}\label{fig:dstats:HMDl:IAI}
\end{figure}
%
%
% Dataset stats: HMDl-IAInorm
\begin{figure}[t]
%\captionsetup[subfigure]{labelformat=empty}
\centering
\subfloat[\Gls{teental}]{\label{fig:dstats:HMDl:IAInorm:teen}\includegraphics[scale=1]{dstats/HMDl-teen-IAInorm.pdf}} \hspace{0.5cm} 
\subfloat[\Gls{ektal}]{\label{fig:dstats:HMDl:IAInorm:ek}\includegraphics[scale=1]{dstats/HMDl-ek-IAInorm.pdf}} \\ 
\subfloat[\Gls{jhaptal}]{\label{fig:dstats:HMDl:IAInorm:jhap}\includegraphics[scale=1]{dstats/HMDl-jhap-IAInorm.pdf}} \hspace{0.5cm} 
\subfloat[\Gls{rupak}]{\label{fig:dstats:HMDl:IAInorm:rupak}\includegraphics[scale=1]{dstats/HMDl-rupak-IAInorm.pdf}} \\ 
\caption[HMDl-IAInorm]{HMDl-IAInorm}\label{fig:dstats:HMDl:IAInorm}
\centering
\end{figure}
%
%
% Dataset stats: HMDl-ISI
\begin{figure}[t]
%\captionsetup[subfigure]{labelformat=empty}
\centering
\subfloat[\Gls{teental}]{\label{fig:dstats:HMDl:ISI:teen}\includegraphics[scale=1]{dstats/HMDl-teen-ISI.pdf}} \hspace{0.5cm} 
\subfloat[\Gls{ektal}]{\label{fig:dstats:HMDl:ISI:ek}\includegraphics[scale=1]{dstats/HMDl-ek-ISI.pdf}} \\ 
\subfloat[\Gls{jhaptal}]{\label{fig:dstats:HMDl:ISI:jhap}\includegraphics[scale=1]{dstats/HMDl-jhap-ISI.pdf}} \hspace{0.5cm} 
\subfloat[\Gls{rupak}]{\label{fig:dstats:HMDl:ISI:rupak}\includegraphics[scale=1]{dstats/HMDl-rupak-ISI.pdf}} \\ 
\caption[HMDl-ISI]{HMDl-ISI}\label{fig:dstats:HMDl:ISI}
\end{figure}
%
%
% Dataset stats: HMDl-ISInorm
\begin{figure}[t]
%\captionsetup[subfigure]{labelformat=empty}
\centering
\subfloat[\Gls{teental}]{\label{fig:dstats:HMDl:ISInorm:teen}\includegraphics[scale=1]{dstats/HMDl-teen-ISInorm.pdf}} \hspace{0.5cm} 
\subfloat[\Gls{ektal}]{\label{fig:dstats:HMDl:ISInorm:ek}\includegraphics[scale=1]{dstats/HMDl-ek-ISInorm.pdf}} \\ 
\subfloat[\Gls{jhaptal}]{\label{fig:dstats:HMDl:ISInorm:jhap}\includegraphics[scale=1]{dstats/HMDl-jhap-ISInorm.pdf}} \hspace{0.5cm} 
\subfloat[\Gls{rupak}]{\label{fig:dstats:HMDl:ISInorm:rupak}\includegraphics[scale=1]{dstats/HMDl-rupak-ISInorm.pdf}} \\ 
\caption[HMDl-ISInorm]{HMDl-ISInorm}\label{fig:dstats:HMDl:ISInorm}
\end{figure}
%
%
%\clearpage
%% Dataset stats: HMDs-IAI
\begin{figure}[ht]
%\captionsetup[subfigure]{labelformat=empty}
\centering
\subfloat[\Gls{teental}]{\label{fig:dstats:HMDs:IAI:teen}\includegraphics[scale=1]{dstats/HMDs-teen-IAI.pdf}} \hspace{0.5cm} 
\subfloat[\Gls{ektal}]{\label{fig:dstats:HMDs:IAI:ek}\includegraphics[scale=1]{dstats/HMDs-ek-IAI.pdf}} \\ 
\subfloat[\Gls{jhaptal}]{\label{fig:dstats:HMDs:IAI:jhap}\includegraphics[scale=1]{dstats/HMDs-jhap-IAI.pdf}} \hspace{0.5cm} 
\subfloat[\Gls{rupak}]{\label{fig:dstats:HMDs:IAI:rupak}\includegraphics[scale=1]{dstats/HMDs-rupak-IAI.pdf}} \\ 
\caption[HMDs-IAI]{HMDs-IAI}\label{fig:dstats:HMDs:IAI}
\end{figure}
%
%
% Dataset stats: HMDs-IAInorm
\begin{figure}[ht]
%\captionsetup[subfigure]{labelformat=empty}
\centering
\subfloat[\Gls{teental}]{\label{fig:dstats:HMDs:IAInorm:teen}\includegraphics[scale=1]{dstats/HMDs-teen-IAInorm.pdf}} \hspace{0.5cm} 
\subfloat[\Gls{ektal}]{\label{fig:dstats:HMDs:IAInorm:ek}\includegraphics[scale=1]{dstats/HMDs-ek-IAInorm.pdf}} \\ 
\subfloat[\Gls{jhaptal}]{\label{fig:dstats:HMDs:IAInorm:jhap}\includegraphics[scale=1]{dstats/HMDs-jhap-IAInorm.pdf}} \hspace{0.5cm} 
\subfloat[\Gls{rupak}]{\label{fig:dstats:HMDs:IAInorm:rupak}\includegraphics[scale=1]{dstats/HMDs-rupak-IAInorm.pdf}} \\ 
\caption[HMDs-IAInorm]{HMDs-IAInorm}\label{fig:dstats:HMDs:IAInorm}
\end{figure}
%
%
% Dataset stats: HMDs-ISI
\begin{figure}[ht]
%\captionsetup[subfigure]{labelformat=empty}
\centering
\subfloat[\Gls{teental}]{\label{fig:dstats:HMDs:ISI:teen}\includegraphics[scale=1]{dstats/HMDs-teen-ISI.pdf}} \hspace{0.5cm} 
\subfloat[\Gls{ektal}]{\label{fig:dstats:HMDs:ISI:ek}\includegraphics[scale=1]{dstats/HMDs-ek-ISI.pdf}} \\ 
\subfloat[\Gls{jhaptal}]{\label{fig:dstats:HMDs:ISI:jhap}\includegraphics[scale=1]{dstats/HMDs-jhap-ISI.pdf}} \hspace{0.5cm} 
\subfloat[\Gls{rupak}]{\label{fig:dstats:HMDs:ISI:rupak}\includegraphics[scale=1]{dstats/HMDs-rupak-ISI.pdf}} \\ 
\caption[HMDs-ISI]{HMDs-ISI}\label{fig:dstats:HMDs:ISI}
\end{figure}
%
%
% Dataset stats: HMDs-ISInorm
\begin{figure}[ht]
%\captionsetup[subfigure]{labelformat=empty}
\centering
\subfloat[\Gls{teental}]{\label{fig:dstats:HMDs:ISInorm:teen}\includegraphics[scale=1]{dstats/HMDs-teen-ISInorm.pdf}} \hspace{0.5cm} 
\subfloat[\Gls{ektal}]{\label{fig:dstats:HMDs:ISInorm:ek}\includegraphics[scale=1]{dstats/HMDs-ek-ISInorm.pdf}} \\ 
\subfloat[\Gls{jhaptal}]{\label{fig:dstats:HMDs:ISInorm:jhap}\includegraphics[scale=1]{dstats/HMDs-jhap-ISInorm.pdf}} \hspace{0.5cm} 
\subfloat[\Gls{rupak}]{\label{fig:dstats:HMDs:ISInorm:rupak}\includegraphics[scale=1]{dstats/HMDs-rupak-ISInorm.pdf}} \\ 
\caption[HMDs-ISInorm]{HMDs-ISInorm}\label{fig:dstats:HMDs:ISInorm}
\end{figure}
%
%
\subsubsection{Rhythm patterns in Hindustani rhythm datasets}
% Dataset stats: HMDs-ISInorm
\begin{figure}[t]
%\captionsetup[subfigure]{labelformat=empty}
\centering
\subfloat[\Gls{teental}]{\label{fig:dstats:HMDs:ISInorm:teen}\includegraphics[scale=0.9]{dstats/HMDs-teen-ISInorm.pdf}} \hspace{0.5cm} 
\subfloat[\Gls{ektal}]{\label{fig:dstats:HMDs:ISInorm:ek}\includegraphics[scale=0.9]{dstats/HMDs-ek-ISInorm.pdf}} \\ 
\subfloat[\Gls{jhaptal}]{\label{fig:dstats:HMDs:ISInorm:jhap}\includegraphics[scale=0.9]{dstats/HMDs-jhap-ISInorm.pdf}} \hspace{0.5cm} 
\subfloat[\Gls{rupak}]{\label{fig:dstats:HMDs:ISInorm:rupak}\includegraphics[scale=0.9]{dstats/HMDs-rupak-ISInorm.pdf}} \\ 
\caption[Histogram of median normalized $\protect\isi$ in the \acrshort{HMDs} dataset]{A histogram of the median normalized inter-\gls{sam} interval $\protect\isi$ in the \acrshort{HMDs} dataset for each \gls{taal}. The ordinate is the fraction of the total count corresponding to the normalized $\isi$ value shown in abscissa.}\label{fig:dstats:HMDs:ISInorm}
\end{figure}
%
% Dataset stats: HMDs-IAInorm
\begin{figure}[t]
%\captionsetup[subfigure]{labelformat=empty}
\centering
\subfloat[\Gls{teental}]{\label{fig:dstats:HMDs:IAInorm:teen}\includegraphics[scale=0.9]{dstats/HMDs-teen-IAInorm.pdf}} \hspace{0.5cm} 
\subfloat[\Gls{ektal}]{\label{fig:dstats:HMDs:IAInorm:ek}\includegraphics[scale=0.9]{dstats/HMDs-ek-IAInorm.pdf}} \\ 
\subfloat[\Gls{jhaptal}]{\label{fig:dstats:HMDs:IAInorm:jhap}\includegraphics[scale=0.9]{dstats/HMDs-jhap-IAInorm.pdf}} \hspace{0.5cm} 
\subfloat[\Gls{rupak}]{\label{fig:dstats:HMDs:IAInorm:rupak}\includegraphics[scale=0.9]{dstats/HMDs-rupak-IAInorm.pdf}} \\ 
\caption[Histogram of median normalized $\protect\ibi$ in the \acrshort{HMDs} dataset]{A histogram of the median normalized inter-\gls{matra} interval $\protect\ibi$ in the \acrshort{HMDs} dataset for each \gls{taal}. The ordinate is the fraction of the total count corresponding to the normalized $\ibi$ value shown in abscissa.}\label{fig:dstats:HMDs:IAInorm}
\end{figure}

% Commented HMDf tala patterns latex code moved to the end
Similar to Carnatic music, we do corpora level analysis of rhythm patterns\index{Rhythm pattern} in Hindustani music and illustrate several musicological inferences and insights, and contrast if there are any differences between music theory and practice. The rhythm patterns described in this section were obtained using spectral flux, in an identical process as described for Carnatic music. % \comment{explain further: what are these differences ?}

The \figrefs{fig:tt:HMDl:teen}{fig:tt:HMDs:rupak}\ show the cycle length rhythm patterns for all \glspl{taal} for both \acrshort{HMDl} and \acrshort{HMDs} datasets, using the spectral flux feature computed identically to the way it was computed for Carnatic music rhythm patterns, as an average over the entire dataset indicated. In each figure, the bottom pane corresponds to the low frequency band ($\protect\obsLow$) and the top pane corresponds to the high frequency band ($\protect\obsHigh$). The abscissa is the \gls{matra} number within the cycle (dotted lines), with 1 indicating the \gls{sam} (marked with a red line). The start of each \gls{vibhaag} is indicated at the top of each pane (\gls{sam} shown as $\times$).

The rhythm patterns in Hindustani are indicative of \gls{tabla} strokes played in the cycle. In the figures, the bottom pane that shows the low frequency band has content from the \gls{bayan} (the left bass drum) of the \gls{tabla} while the top pane has content predominantly from the \gls{dayan} (the right pitched drum) of the \gls{tabla}, but additionally from the lead melody. Hence, for the purpose of this discussion, we use the terms left and right accents to refer to the accents in rhythm patterns from the bottom and top pane, respectively. 

The left and right accents provide interesting insights into the patterns played within a \gls{taal} cycle. We additionally compare rhythm patterns across the \glspl{lay} by plotting the patterns for \acrshort{HMDl} dataset (with \gls{vilambit} \gls{lay} pieces) and \acrshort{HMDs} dataset (\gls{madhyam} and \gls{dhrut} lay pieces) - for each \gls{taal}, the patterns for these two data subsets are plotted in two figures one below the other.

The patterns played in a \gls{taal} cycle have both energy/amplitude accents due to varying strength of the \gls{tabla} stroke and also timbral characteristics, due to the specific stroke played. The rhythm patterns have been generated using the spectral flux feature, which models mostly only energy, and hence can only explain energy accents with these figures. We list down and discuss some salient qualitative observations from the figures for each \gls{taal}, for both \gls{vilambit} \gls{lay} and \gls{madhyam}/\gls{dhrut} \gls{lay}. The patterns are indicative of the surface rhythm present in these audio recordings.

There are several observations from the plotted rhythm patterns that have interesting musicological significance. A professional Hindustani musician has informally validated these observations, but they still have to be formally studied in depth to make valid musicological conclusions. Overall, from \figrefs{fig:tt:HMDl:teen}{fig:tt:HMDs:rupak}, we observe across all \glspl{taal} and \glspl{lay} that accents are stronger on the \glspl{matra}, with accents present even at half and fourth divisions of the matra in many cases. The \gls{sam} most often has the strongest accent. Unlike Carnatic \glspl{tala}, \glspl{theka} in Hindustani music are less flexible, and hence we can infer several concrete conclusions from the rhythm patterns of Hindustani music. 

Across all \glspl{taal} in \gls{vilambit} \gls{lay}, we see additional filler strokes present between \glspl{matra}, showing that percussionists add further metrical subdivisions lower than the \gls{matra}, though not defined in theory. These fillers are also mostly concentrated towards the second half of the \gls{matra}. The 1\tsup{st} \gls{matra} (and often the 2\tsup{nd} \gls{matra}) is quite empty with few accents, while the last few \glspl{matra} of the cycle have dense accents. This is to place a special emphasis on the \gls{sam}, indicating the approaching of \gls{sam} with fillers and dense stroke playing, while there is a short recovery period after the \gls{sam} with fewer strokes. In addition, a dense matra with many fillers is often followed by a sparsely accented \gls{matra} to better contrast the progression through the \gls{taal} cycle, e.g. a dense \gls{matra} 9 after a quieter \gls{matra} 8 in \figref{fig:tt:HMDl:teen}. %and \gls{madhyam} 

Due to the large \gls{matra} period ($\ibi$) in \gls{vilambit} \gls{lay}, each \gls{matra} acts as an anchor for timekeeping, and can be played without any effect from the previous strokes (in fast \gls{tabla} playing in \gls{dhrut}, the previous stroke can possibly affect the sound, intonation, and playing technique of the following strokes). Further, due to a large time interval available to play the \gls{theka}, the \gls{tabla} playing musician focuses on modulation of left bass strokes that can sustain longer. Finally, left and right hand can operate independently, which means modulation of accents through the cycle can be different for left and right accents. The left and right strokes also complement each other. Each of these effects can be observed in the patterns of \gls{vilambit} \gls{lay}. % 

In contrast, across all \glspl{taal} in \gls{madhyam} and \gls{dhrut} \gls{lay}, given the shorter cycles, we see that \glspl{vibhaag} are anchors. The fillers are largely restricted only to half \gls{matra}, with lower accents. \Gls{dhrut} pieces also have a relatively more relaxed timing, and the focus is on right strokes, with the left hand playing the theory defined ``textbook” strokes for timekeeping. In addition, the left and right hands are in sync, which can be seen in the modulation of accents through the cycle being well correlated for both left and right accents - the left and right strokes work together here, in contrast to complementing each other as in \gls{vilambit} \gls{lay}. Furthermore, the patterns differ widely between the \gls{lay} classes, especially for \gls{ektal} and \gls{teental}. 

We now present some \gls{taal} specific observations from the rhythm patterns for each \gls{taal}. Some of these observations corroborate the theory while some of them show the contrast between theory and practice. These inferences mainly address \gls{tabla} stroke playing during the cycles, while the effects of melody has not been considered into account. This is a valid assumption to make since these patterns are averaged over several cycles, averaging out and reducing the effect of melody on these rhythm patterns. 
% Tala pattern: HMDl-teen-all-lo230-superflux-mvavg-normZ, HMDl-teen-all-hi250-superflux-mvavg-normZ
\begin{figure}[t]
\captionsetup[subfigure]{labelformat=empty}
\centering
\subfloat[]{\label{fig:tt:HMDl:teen:hi}\includegraphics[width=\textwidth]{talaPatts/HMDl-teen-all-hi250-superflux-mvavg-normZ.pdf}} \\ \vspace{-1.35cm}
\subfloat[]{\label{fig:tt:HMDl:teen:lo}\includegraphics[width=\textwidth]{talaPatts/HMDl-teen-all-lo230-superflux-mvavg-normZ.pdf}}
\caption[Rhythm patterns in \gls{teental} learned from \acrshort{HMDl} dataset]{Cycle length rhythmic patterns learned from \acrshort{HMDl} dataset for \gls{teental}, computed from spectral flux feature and averaged over all the pieces in the dataset. The bottom/top pane corresponds to the low/high frequency bands, respectively. The abscissa is the \gls{matra} number within the cycle (dotted lines), with 1 indicating the \gls{sam} (marked with a red line). The start of each \gls{vibhaag} is indicated at the top of each pane (\gls{sam} shown as $\times$). The plot shows the cycle extended by a \gls{matra} at the beginning and end to illustrate the cyclic nature of the \gls{taal}.}\label{fig:tt:HMDl:teen} % HMDl-teen-all-hi250-superflux-mvavg-normZ,HMDl-teen-all-lo230-superflux-mvavg-normZ
\end{figure}
% , computed from spectral flux feature and averaged over all the pieces in the dataset. The bottom/top pane corresponds to the low/high frequency bands, respectively. The abscissa is the \gls{matra} number within the cycle (dotted lines), with 1 indicating the \gls{sam} (marked with a red line). The start of each \gls{vibhaag} is indicated at the top of each pane (\gls{sam} shown as $\times$).
% 
%
% Tala pattern: HMDs-teen-all-lo230-superflux-mvavg-normZ, HMDs-teen-all-hi250-superflux-mvavg-normZ
\begin{figure}[t]
\captionsetup[subfigure]{labelformat=empty}
\centering
\subfloat[]{\label{fig:tt:HMDs:teen:hi}\includegraphics[width=\textwidth]{talaPatts/HMDs-teen-all-hi250-superflux-mvavg-normZ.pdf}} \\ \vspace{-1.35cm}
\subfloat[]{\label{fig:tt:HMDs:teen:lo}\includegraphics[width=\textwidth]{talaPatts/HMDs-teen-all-lo230-superflux-mvavg-normZ.pdf}}
\caption[Rhythm patterns in \gls{teental} learned from \acrshort{HMDs} dataset]{Cycle length rhythmic patterns learned from \acrshort{HMDs} dataset for \gls{teental}.}\label{fig:tt:HMDs:teen}
\end{figure}
%
%
% From \figrefs{fig:tt:HMDl:teen}{fig:tt:HMDs:teen}
\begin{description}[style=unboxed,leftmargin=0cm]
\item[\textbf{\Gls{vilambit} \gls{teental}}:] From \figref{fig:tt:HMDl:teen}, we see that the 14\tsup{th} matra has the strongest left accent, and the last \gls{matra} (matra 16) has many fillers, both to indicate the arrival of \gls{sam} - a phenomenon known in music theory as \gls{amad} (literal meaning - the approach). A strong left accent on the 9\tsup{th} matra is not defined in theory (the stroke in the \gls{theka} is a right stroke \syl{NA}), but often a \syl{DHA} is played instead. This is a known (to practising musicians) difference between theory and practice and can additionally be observed in the patterns too. As described earlier, the right stroke fillers are fewer in \glspl{matra} 1 and 2, and the left accents support the timekeeping task when the right accents are weaker there. 4\tsup{th} \gls{matra} has a strong right accent perhaps to indicate the end of the 1\tsup{st} \gls{vibhaag}, after a filler-less \glspl{matra} 2 and 3. The beginning of the 2\tsup{nd} and 3\tsup{rd} \glspl{vibhaag}, labeled 2 and 0 have higher number of fillers. The left accents between the 11\tsup{th} and the 14\tsup{th} matra are weak - with the 11\tsup{th} and 14\tsup{th} \gls{matra} accents acting as anchors for the ``quiet" created in between them. It is interesting to note the varying modulation of accent levels through the \glspl{vibhaag} of the cycle. Specifically, we can see that the left and right accent envelopes through the cycle are complementary, indicating that left and right drums are complementary in \gls{vilambit} \gls{lay}. % 5th matra has complementary left and right accents, stronger second filler left accent compared to stronger first filler left accent - why ? (Many a places the bol "Dha - Dha Ge" is played where the second filler 'Ge' is a only-left stroke resulting in relatively higher accent (as the 'Dha' preceding to that also has a right accent))
%
\item[\textbf{\Gls{madhyam} and \gls{dhrut} \gls{lay} \gls{teental}}:] From \figref{fig:tt:HMDs:teen}, we see that the filler strokes in \gls{dhrut} \gls{teental} are restricted to a single filler at half \gls{matra} positions in contrast to three of more fillers in \gls{vilambit}. The accents are more regular due to higher tempi associated. Similar to \gls{vilambit}, the 9\tsup{th} matra has a strong left accent, which again is a well known difference between theory and practice. The 11\tsup{th} and 14\tsup{th} \glspl{matra} have strong left accents to support the build up of accents through \glspl{matra} 12-14 and indicate the arrival of sam (\gls{amad}). It is interesting to note that the right accent at \gls{vibhaag} boundary (\gls{matra} 13) is weaker than that at the previous \gls{matra} 12. This is perhaps due to the stroke on \gls{matra} 13 being skipped and a strong left stroke on \gls{matra} 14 often played to indicate the approaching \gls{sam}. 
%\end{description}
%\begin{description}[style=unboxed,leftmargin=0cm]
\item[\textbf{\Gls{vilambit} \gls{ektal}}:] From \figref{fig:tt:HMDl:ek}, we see that the last matra of the cycle before the \gls{sam} (\gls{matra} 12) has dense accents, with the final filler strokes having stronger left accents than the \gls{sam}. This is another example of \gls{amad}, where the approach of a \gls{sam} is distinctly indicated. The \glspl{matra} 4 and 10 (both with the \gls{theka} \gls{bol} \syl{TI} \syl{RA} \syl{KI} \syl{TA}, see \tabref{fig:theka:hindustani}) have equal accents in theory. However, \gls{matra} 10 has stronger accents than 4 in practice since it is closer to the \gls{sam}. \syl{TI} \syl{RA} \syl{KI} \syl{TA} is often played with more than four strokes towards the end of the matra 4 and 10. Since \syl{TI} \syl{RA} \syl{KI} \syl{TA} is dense, the \gls{matra} following them (\glspl{matra} 5 and 11) have less fillers. In addition, only \glspl{matra} 4 and 10 have fillers distributed throughout the \gls{matra}, while the rest have fillers only towards the end. \Glspl{vibhaag} 2 and 3 (spanning \glspl{matra} 3-6) and \glspl{vibhaag} 5 and 6 (spanning \gls{matra} 9-$\times$) are identical in theory, but we can see several deviations in performance, with \glspl{vibhaag} 5 and 6 having stronger left accents since they are closer to \gls{sam}. Further, the strokes \syl{DHIN} at \gls{matra} 1 and \gls{matra} 2 are identical in theory, but in practice the \syl{DHIN} at \gls{matra} 2 is played softer to differentiate it from the \syl{DHIN} at the \gls{sam}. The modulation of right accent levels through the cycle is interesting, with stronger accents occurring when the \gls{matra} is less dense with lesser number of accents. This has a functional role in timekeeping - aided by stronger accents and denser \glspl{matra}, which complement each other. 
\end{description}
%
% Tala pattern: HMDl-ek-all-lo230-superflux-mvavg-normZ, HMDl-ek-all-hi250-superflux-mvavg-normZ
\begin{figure}[t]
\captionsetup[subfigure]{labelformat=empty}
\centering
\subfloat[]{\label{fig:tt:HMDl:ek:hi}\includegraphics[width=\textwidth]{talaPatts/HMDl-ek-all-hi250-superflux-mvavg-normZ.pdf}} \\ \vspace{-1.35cm}
\subfloat[]{\label{fig:tt:HMDl:ek:lo}\includegraphics[width=\textwidth]{talaPatts/HMDl-ek-all-lo230-superflux-mvavg-normZ.pdf}}
\caption[Rhythm patterns in \gls{ektal} learned from \acrshort{HMDl} dataset]{Cycle length rhythmic patterns learned from \acrshort{HMDl} dataset for \gls{ektal}.}\label{fig:tt:HMDl:ek}
\end{figure}
%
% Tala pattern: HMDs-ek-all-lo230-superflux-mvavg-normZ, HMDs-ek-all-hi250-superflux-mvavg-normZ
\begin{figure}[t]
\captionsetup[subfigure]{labelformat=empty}
\centering
\subfloat[]{\label{fig:tt:HMDs:ek:hi}\includegraphics[width=\textwidth]{talaPatts/HMDs-ek-all-hi250-superflux-mvavg-normZ.pdf}} \\ \vspace{-1.35cm}
\subfloat[]{\label{fig:tt:HMDs:ek:lo}\includegraphics[width=\textwidth]{talaPatts/HMDs-ek-all-lo230-superflux-mvavg-normZ.pdf}}
\caption[Rhythm patterns in \gls{ektal} learned from \acrshort{HMDs} dataset]{Cycle length rhythmic patterns learned from \acrshort{HMDs} dataset for \gls{ektal}.}\label{fig:tt:HMDs:ek}
\end{figure}
%
%
% Tala pattern: HMDl-jhap-all-lo230-superflux-mvavg-normZ, HMDl-jhap-all-hi250-superflux-mvavg-normZ
\begin{figure}[t]
\captionsetup[subfigure]{labelformat=empty}
\centering
\subfloat[]{\label{fig:tt:HMDl:jhap:hi}\includegraphics[width=\textwidth]{talaPatts/HMDl-jhap-all-hi250-superflux-mvavg-normZ.pdf}} \\ \vspace{-1.35cm}
\subfloat[]{\label{fig:tt:HMDl:jhap:lo}\includegraphics[width=\textwidth]{talaPatts/HMDl-jhap-all-lo230-superflux-mvavg-normZ.pdf}}
\caption[Rhythm patterns in \gls{jhaptal} learned from \acrshort{HMDl} dataset]{Cycle length rhythmic patterns learned from \acrshort{HMDl} dataset for \gls{jhaptal}.}\label{fig:tt:HMDl:jhap}
\end{figure}
%
% Tala pattern: HMDs-jhap-all-lo230-superflux-mvavg-normZ, HMDs-jhap-all-hi250-superflux-mvavg-normZ
\begin{figure}[t]
\captionsetup[subfigure]{labelformat=empty}
\centering
\subfloat[]{\label{fig:tt:HMDs:jhap:hi}\includegraphics[width=\textwidth]{talaPatts/HMDs-jhap-all-hi250-superflux-mvavg-normZ.pdf}} \\ \vspace{-1.35cm}
\subfloat[]{\label{fig:tt:HMDs:jhap:lo}\includegraphics[width=\textwidth]{talaPatts/HMDs-jhap-all-lo230-superflux-mvavg-normZ.pdf}}
\caption[Rhythm patterns in \gls{jhaptal} learned from \acrshort{HMDs} dataset]{Cycle length rhythmic patterns learned from \acrshort{HMDs} dataset for \gls{jhaptal}.}\label{fig:tt:HMDs:jhap}
\centering
\end{figure}
%
\begin{description}[style=unboxed,leftmargin=0cm]
\item[\textbf{\Gls{madhyam} and \gls{dhrut} \gls{lay} \gls{ektal}}:] Though defined with six \glspl{vibhaag} in theory, \gls{dhrut} \gls{ektal} is described better as having four \glspl{vibhaag} of 3 \glspl{matra} each, as shown in \figref{fig:taal:drutektal}, with the \glspl{vibhaag} starting at \glspl{matra} 1, 4, 7, and 10. As can be seen from \figref{fig:tt:HMDs:ek}, the strong right accents due to \syl{NA} stroke at \glspl{matra} 3, 6, 9 and 12 are distinctly seen. This suggests that for \gls{dhrut} \gls{lay}, timekeeping is done more with the sharp right strokes (e.g. `\syl{NA}' here) and accentuation can even be at non-\gls{vibhaag} marker \glspl{matra} such as 6 and 12. Even though the last \gls{vibhaag} starts on matra 10, there is strong right accent on matra 9, an indication of the approaching \gls{sam} (\gls{amad}). The four strokes in \syl{TI} \syl{RA} \syl{KI} \syl{TA} is often not played in \gls{dhrut}, replacing it with just two strokes \syl{TE} \syl{KE} - we see only two accents in \glspl{matra} 4 and 10. In addition, due to the dense stroke playing on \gls{matra} 4 and 10, the left accents in \gls{matra} 6 and 12 are quiet with relatively weaker accents. Similar to \gls{vilambit} \gls{ektal}, though the first and second matra have equal accented \syl{DHIN} stroke in theory, \syl{DHIN} on the second \gls{matra} is played considerably softer with weak accent. As with all \glspl{taal} in \gls{dhrut} \gls{lay}, the accents on left and right through the cycle are correlated. 
\end{description}
%
% Tala pattern: HMDl-rupak-all-lo230-superflux-mvavg-normZ, HMDl-rupak-all-hi250-superflux-mvavg-normZ
\begin{figure}[t]
\captionsetup[subfigure]{labelformat=empty}
\centering
\subfloat[]{\label{fig:tt:HMDl:rupak:hi}\includegraphics[width=\textwidth]{talaPatts/HMDl-rupak-all-hi250-superflux-mvavg-normZ.pdf}} \\ \vspace{-1.35cm}
\subfloat[]{\label{fig:tt:HMDl:rupak:lo}\includegraphics[width=\textwidth]{talaPatts/HMDl-rupak-all-lo230-superflux-mvavg-normZ.pdf}}
\caption[Rhythm patterns in \gls{rupak} learned from \acrshort{HMDl} dataset]{Cycle length rhythmic patterns learned from \acrshort{HMDl} dataset for \gls{rupak}.}\label{fig:tt:HMDl:rupak}
\end{figure}
%
%
% Tala pattern: HMDs-rupak-all-lo230-superflux-mvavg-normZ, HMDs-rupak-all-hi250-superflux-mvavg-normZ
\begin{figure}[t]
\captionsetup[subfigure]{labelformat=empty}
\centering
\subfloat[]{\label{fig:tt:HMDs:rupak:hi}\includegraphics[width=\textwidth]{talaPatts/HMDs-rupak-all-hi250-superflux-mvavg-normZ.pdf}} \\ \vspace{-1.35cm}
\subfloat[]{\label{fig:tt:HMDs:rupak:lo}\includegraphics[width=\textwidth]{talaPatts/HMDs-rupak-all-lo230-superflux-mvavg-normZ.pdf}}
\caption[Rhythm patterns in \gls{rupak} learned from \acrshort{HMDs} dataset]{Cycle length rhythmic patterns learned from \acrshort{HMDs} dataset for \gls{rupak}.}\label{fig:tt:HMDs:rupak}
\end{figure}
%
\begin{description}[style=unboxed,leftmargin=0cm]
\item[\textbf{\Gls{vilambit} \gls{jhaptal}}:] From \figref{fig:tt:HMDl:jhap}, we see that all the \syl{NA} strokes (\glspl{matra} 2, 5, 7, 10) have a strong right accent and weak left accents, as described in theory. There are filler strokes to end the \glspl{vibhaag} at \glspl{matra} 2 and 7. This can be explained with the often played variant of the \gls{jhaptal} \gls{theka} (\syl{DHI} \syl{NA-TE-KE} \syl{DHI} \syl{DHI} \syl{NA} | \syl{TI} \syl{NA-TE-KE} \syl{DHI} \syl{DHI} \syl{NA}). There are further strong accented fillers on \glspl{matra} 5 and 10 that act as anchor points to indicate the end of half and full cycle. 
%
\item[\textbf{\Gls{madhyam} and \gls{dhrut} \gls{lay} \gls{jhaptal}}:] \figref{fig:tt:HMDs:jhap} shows that the left accents are as defined in theory with basic \gls{theka} playing. The envelope of accents through the cycle is more regular than in \gls{vilambit} \gls{jhaptal}. In theory, the \gls{vibhaag} 2 (\glspl{matra} 3-5) and \gls{vibhaag} 4 (\glspl{matra} 8-10) are identical, but some deviations can be observed in practice. % (a ``textbook" \gls{bayan} playing)
\end{description}
%
\begin{description}[style=unboxed,leftmargin=0cm]
\item[\textbf{\Gls{vilambit} \gls{rupak}}:] \Gls{rupak} is defined in theory with no left accents on \glspl{matra} 1 and 2, but in practice left strokes are often played (with closed strokes than modulated sustained left strokes). This also implies that \gls{rupak} having a \gls{khali} (0) on the \gls{sam} does not mean it is less accented. \Gls{rupak} is defined to have a 3+2+2 structure, but we see from \figref{fig:tt:HMDl:rupak} that \gls{matra} 2 has a strong left accent, which acts as an anchor, giving the \gls{vilambit} \gls{rupak} a 1+2+2+2 structure, which is close to the tapping of \gls{mishra chapu} \gls{tala} of Carnatic music in practice. This could also be because musicians might play with the same accent on both \syl{TIN} (\glspl{matra} 1 and 2) with a \syl{KAT} stroke to contrast with the \syl{NA} stoke which is less left-accented. The \gls{vibhaag} 2 (\glspl{matra} 4-5) and \gls{vibhaag} 3 (\gls{matra} 6-7) are identical in theory, but in practice the accents differ. \Gls{matra} 5 has the strongest right accent (\syl{NA} stroke), perhaps indicating \gls{amad}. Fillers are more on \gls{matra} 3, to end \gls{vibhaag} 1. In general, we also see that the fillers get more dense towards the end of \glspl{vibhaag}. 
%
\item[\textbf{\Gls{madhyam} and \gls{dhrut} \gls{lay} \gls{rupak}}:] From \figref{fig:tt:HMDs:rupak}, the left strokes and accents closely follow the description in theory. The strongest left accent is on \gls{matra} 4, as defined in theory. The \gls{vibhaag} 2 and 3 are identical with similar accents. Interestingly, the fillers grow through the cycle, becoming more dense towards the end of the cycle. In \gls{dhrut} \gls{rupak}, the accent on the second \gls{matra} is softer than \gls{vilambit} \gls{rupak}, going back to its canonical 3+2+2 structure compared to 1+2+2+2 structure in \gls{vilambit} \gls{rupak}. 
\end{description}
%
% 
%%%%%%%%%%%%%%%%%%%%%%%%%%%%%%%%%%% HMDf tala patterns moved here %%%%%%%%%%%%%%%%%%%%%%%%%
% Tala pattern: HMDf-teen-all-lo230-superflux-mvavg-normZ, HMDf-teen-all-hi250-superflux-mvavg-normZ
% \begin{figure}
% \captionsetup[subfigure]{labelformat=empty}
% \centering
% \subfloat[]{\label{fig:tt:HMDf:teen:hi}\includegraphics[width=\textwidth]{talaPatts/HMDf-teen-all-hi250-superflux-mvavg-normZ.pdf}} \\ \vspace{-1.35cm}
% \subfloat[]{\label{fig:tt:HMDf:teen:lo}\includegraphics[width=\textwidth]{talaPatts/HMDf-teen-all-lo230-superflux-mvavg-normZ.pdf}}
% \caption[HMDf-teen]{HMDf-teen-all-hi250-superflux-mvavg-normZ,HMDf-teen-all-lo230-superflux-mvavg-normZ}\label{fig:tt:HMDf:teen}
% \end{figure}
%
%
% Tala pattern: HMDf-ek-all-lo230-superflux-mvavg-normZ, HMDf-ek-all-hi250-superflux-mvavg-normZ
% \begin{figure}
% \captionsetup[subfigure]{labelformat=empty}
% \centering
% \subfloat[]{\label{fig:tt:HMDf:ek:hi}\includegraphics[width=\textwidth]{talaPatts/HMDf-ek-all-hi250-superflux-mvavg-normZ.pdf}} \\ \vspace{-1.35cm}
% \subfloat[]{\label{fig:tt:HMDf:ek:lo}\includegraphics[width=\textwidth]{talaPatts/HMDf-ek-all-lo230-superflux-mvavg-normZ.pdf}}
% \caption[HMDf-ek]{HMDf-ek-all-hi250-superflux-mvavg-normZ,HMDf-ek-all-lo230-superflux-mvavg-normZ}\label{fig:tt:HMDf:ek}
% \end{figure}
%
%
% Tala pattern: HMDf-jhap-all-lo230-superflux-mvavg-normZ, HMDf-jhap-all-hi250-superflux-mvavg-normZ
% \begin{figure}
% \captionsetup[subfigure]{labelformat=empty}
% \centering
% \subfloat[]{\label{fig:tt:HMDf:jhap:hi}\includegraphics[width=\textwidth]{talaPatts/HMDf-jhap-all-hi250-superflux-mvavg-normZ.pdf}} \\ \vspace{-1.35cm}
% \subfloat[]{\label{fig:tt:HMDf:jhap:lo}\includegraphics[width=\textwidth]{talaPatts/HMDf-jhap-all-lo230-superflux-mvavg-normZ.pdf}}
% \caption[HMDf-jhap]{HMDf-jhap-all-hi250-superflux-mvavg-normZ,HMDf-jhap-all-lo230-superflux-mvavg-normZ}\label{fig:tt:HMDf:jhap}
% \end{figure}
%
%
% Tala pattern: HMDf-rupak-all-lo230-superflux-mvavg-normZ, HMDf-rupak-all-hi250-superflux-mvavg-normZ
% \begin{figure}
% \captionsetup[subfigure]{labelformat=empty}
% \centering
% \subfloat[]{\label{fig:tt:HMDf:rupak:hi}\includegraphics[width=\textwidth]{talaPatts/HMDf-rupak-all-hi250-superflux-mvavg-normZ.pdf}} \\ \vspace{-1.35cm}
% \subfloat[]{\label{fig:tt:HMDf:rupak:lo}\includegraphics[width=\textwidth]{talaPatts/HMDf-rupak-all-lo230-superflux-mvavg-normZ.pdf}}
% \caption[HMD-rupak]{HMDf-rupak-all-hi250-superflux-mvavg-normZ,HMDf-rupak-all-lo230-superflux-mvavg-normZ}\label{fig:tt:HMDf:rupak}
% \end{figure}
%
% \clearpage  
% % Dataset stats: HMDf-IAI
\begin{figure}[ht]
%\captionsetup[subfigure]{labelformat=empty}
\begin{center}
\subfloat[\Gls{teental}]{\label{fig:dstats:HMDf:IAI:teen}\includegraphics[scale=1]{dstats/HMDf-teen-IAI.pdf}} \hspace{0.5cm} 
\subfloat[\Gls{ektal}]{\label{fig:dstats:HMDf:IAI:ek}\includegraphics[scale=1]{dstats/HMDf-ek-IAI.pdf}} \\ 
\subfloat[\Gls{jhaptal}]{\label{fig:dstats:HMDf:IAI:jhap}\includegraphics[scale=1]{dstats/HMDf-jhap-IAI.pdf}} \hspace{0.5cm} 
\subfloat[\Gls{rupak}]{\label{fig:dstats:HMDf:IAI:rupak}\includegraphics[scale=1]{dstats/HMDf-rupak-IAI.pdf}} \\ 
\caption[HMDf-IAI]{HMDf-IAI}\label{fig:dstats:HMDf:IAI}
\end{center}
\end{figure}


% Dataset stats: HMDf-IAInorm
\begin{figure}[ht]
%\captionsetup[subfigure]{labelformat=empty}
\begin{center}
\subfloat[\Gls{teental}]{\label{fig:dstats:HMDf:IAInorm:teen}\includegraphics[scale=1]{dstats/HMDf-teen-IAInorm.pdf}} \hspace{0.5cm} 
\subfloat[\Gls{ektal}]{\label{fig:dstats:HMDf:IAInorm:ek}\includegraphics[scale=1]{dstats/HMDf-ek-IAInorm.pdf}} \\ 
\subfloat[\Gls{jhaptal}]{\label{fig:dstats:HMDf:IAInorm:jhap}\includegraphics[scale=1]{dstats/HMDf-jhap-IAInorm.pdf}} \hspace{0.5cm} 
\subfloat[\Gls{rupak}]{\label{fig:dstats:HMDf:IAInorm:rupak}\includegraphics[scale=1]{dstats/HMDf-rupak-IAInorm.pdf}} \\ 
\caption[HMDf-IAInorm]{HMDf-IAInorm}\label{fig:dstats:HMDf:IAInorm}
\end{center}
\end{figure}


% Dataset stats: HMDf-ISI
\begin{figure}[ht]
%\captionsetup[subfigure]{labelformat=empty}
\begin{center}
\subfloat[\Gls{teental}]{\label{fig:dstats:HMDf:ISI:teen}\includegraphics[scale=1]{dstats/HMDf-teen-ISI.pdf}} \hspace{0.5cm} 
\subfloat[\Gls{ektal}]{\label{fig:dstats:HMDf:ISI:ek}\includegraphics[scale=1]{dstats/HMDf-ek-ISI.pdf}} \\ 
\subfloat[\Gls{jhaptal}]{\label{fig:dstats:HMDf:ISI:jhap}\includegraphics[scale=1]{dstats/HMDf-jhap-ISI.pdf}} \hspace{0.5cm} 
\subfloat[\Gls{rupak}]{\label{fig:dstats:HMDf:ISI:rupak}\includegraphics[scale=1]{dstats/HMDf-rupak-ISI.pdf}} \\ 
\caption[HMDf-ISI]{HMDf-ISI}\label{fig:dstats:HMDf:ISI}
\end{center}
\end{figure}


% Dataset stats: HMDf-ISInorm
\begin{figure}[ht]
%\captionsetup[subfigure]{labelformat=empty}
\begin{center}
\subfloat[\Gls{teental}]{\label{fig:dstats:HMDf:ISInorm:teen}\includegraphics[scale=1]{dstats/HMDf-teen-ISInorm.pdf}} \hspace{0.5cm} 
\subfloat[\Gls{ektal}]{\label{fig:dstats:HMDf:ISInorm:ek}\includegraphics[scale=1]{dstats/HMDf-ek-ISInorm.pdf}} \\ 
\subfloat[\Gls{jhaptal}]{\label{fig:dstats:HMDf:ISInorm:jhap}\includegraphics[scale=1]{dstats/HMDf-jhap-ISInorm.pdf}} \hspace{0.5cm} 
\subfloat[\Gls{rupak}]{\label{fig:dstats:HMDf:ISInorm:rupak}\includegraphics[scale=1]{dstats/HMDf-rupak-ISInorm.pdf}} \\ 
\caption[HMDf-ISInorm]{HMDf-ISInorm}\label{fig:dstats:HMDf:ISInorm}
\end{center}
\end{figure}


\subsubsection{Applications of the \acrshort{HMDf} dataset}
The \acrshort{HMDf} dataset and its subsets \acrshort{HMDl} and \acrshort{HMDs} datasets are intended to be test datasets for several automatic rhythm analysis tasks in Hindustani music. Possible tasks where the datasets can be used include \gls{sam} and \gls{matra} tracking, tempo estimation and tracking, \gls{taal} recognition, rhythm based segmentation of musical audio, audio to score/lyrics alignment, and rhythmic pattern discovery. In this thesis, these datasets are primarily used for rhythmic pattern analysis and meter inference/tracking. Most of the research results are presented on the two subsets separately, to contrast performance of algorithms across different \gls{lay}. 