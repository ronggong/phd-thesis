\chapter*{Acknowledgements}
\vspace*{-1cm}
% To be completed ... 
%Thank CompMusic 
%
%CompMusic partners. 
%
%Help with datasets. 
%
%Help with code. 
%
%Help with music concepts. 
%
%Help with conferences. 
% We are thankful to Dr. R. Reigle for inspiring discussion and help regarding terminology. We would also like to thank Dr. A. Klapuri, Dr. A. Pikrakis, Dr. M. E. P. Davies, J. Hockman, and S. Gulati providing their code and implementation of the algorithms for evaluation in this paper. This work is partly supported by the European Research Council under the European Union's Seventh Framework Program, as part of the CompMusic project (ERC grant agreement 267583).
Someone said four years is a long time, but for me it has gone past in a blink and has left me wanting more! The main reason I believe for that is the group of people I have been fortunate to be a part of. I am extremely grateful to my advisor Xavier Serra for the direction, guidance, mentoring, support and freedom he has given me. His research vision, initiative, perspective and far-sightedness are some qualities that leave me in awe every time. I also thank him for his initiative of conceiving and leading the CompMusic project, and the opportunity he gave me to be a part of it. 

I wish to express a deep sense of gratitude to Andre Holzapfel, who has guided me throughout our continuous collaboration. I have greatly benefited from his multidisciplinary experience and his deep understanding of rhythm in music cultures around the world. It is a great joy to work with him and learn from his meticulous approach to research spanning diverse areas. When I was stuck in a research problem, he would surely know what was to be done next! His second PhD is a sign of his steadfastness and is a great inspiration for me. I would like to thank Taylan Cemgil for the crucially important ideas stemming from his vast experience with Bayesian methods and inference. I also thank Taylan Cemgil and Andre Holzapfel for inviting me for a short research stay at Boğaziçi University to work with them. % His second PhD is beyond what I consider is humanly possible and is a great inspiration for me!

I thank Hema Murthy and Preeti Rao for their pioneering contributions in CompMusic, and for their guidance, support and collaboration on several research topics. I also thank Barış Bozkurt for his ideas and fruitful discussions. I sincerely thank all my collaborators and co-authors Mark Sandler, Mi Tian, Manoj Kumar and Harshavardhan for their contribution. Special thanks to Martin Clayton for his interest in technology tools for rhythm analysis, guidance on musicological aspects of the work, and a review of parts of the dissertation to provide musicological insights.

I have had the best experience working with the CompMusic team, my culturally diverse research family that supported me both professionally and personally. My personal thanks go to Sankalp Gulati (the dependable guy next seat who can make even a block of wood sound good and even a lump of salt taste good), Gopala Krishna Koduri (the guru of discipline and planning), Sertan Şentürk (Otomobil), Vignesh Ishwar (cooking and singing, up for both), Kaustuv Kanti Ganguli (best Kedar and Kesaribhath ever), Rafael Caro (who spreads laughter contagiously), Swapnil Gupta (the fellow \gls{taal} companion), Mohamed Sordo (the walking-talking experience), Rong Gong (chef de cuisine), Georgi Dzhambazov (the fellow graphical models explorer), Alastair Porter (if he doesn't know, nobody else will), Andres Ferraro, Yile Yang, Vinutha Prasad, Ashwin Bellur, Shuo Zhang, Shrey Dutta, Padi Sarala, Hasan Sercan Atlı, Burak Uyar, Joe Cheri Ross, Sridharan Sankaran, Akshay Anantapadmanabhan, Jom Kuriakose, Jilt Sebastian and Amruta Vidwans. I also thank the collaborators of CompMusic T M Krishna, Amin Chachoo, Suvarnalata Rao for helping us formulate research problems and correcting us with feedback. 

MTG has been an incubator for several of my thoughts and ideas, fueled by encouraging conversations and discussions with many of its past and present members: Sergi Jordà, Emilia Gómez, Rafael Ramírez, Jordi Bonada, Perfecto Herrera, Agustín Martorell, Marius Miron, Oscar Mayor, Frederic Font, Sebastián Mealla, Panos Papiotis, Sergio Giraldo, Sergio Oramas, Jordi Pons, Dmitry Bogdanov, Esteban Maestre, Carles Julià, Juan José Bosch, Nadine Kroher, Cárthach Ó Nuanáin, Álvaro Sarasúa, Zacharias Vamvakousis, Julio Carabias, Giuseppe Bandiera, Martí Umbert and Daniel Gomez. Many thanks to all mentioned here and others I might have missed. Special thanks to Rafael and Jordi for their help in translating the abstract and showing how much my Catalan and Spanish can be improved! Thanks to Cristina Garrido, Sonia Espí, Alba Rosado, Vanessa Jimenez, Jana Safrankova and Lydia García for their reassuring support in hooping through piles of paperwork. 

I gratefully acknowledge Florian Krebs, Sebastian Böck and Gerhard Widmer for their inputs on this work and for providing me access to their code and data. Special thanks to Juan Bello and Carlos Guedes for the opportunity to attend the rhythm workshop at NYUAD (2014) and broaden my horizons with a multitude of perspectives on rhythm. On my journey so far, I have interacted and learned from several more researchers who have been eager to give their inputs and share ideas - (ordered alphabetically by last name) Jean-Julian Aucouturier, Emmanouil Benetos, Matthew Davies, Simon Dixon, Simon Durand, Gyorgy Fazekas, Shayan Garani, Fabien Gouyon, Shantala Hegde, Robert Kaye, Andres Lewin-Richter, Meinard Müller, Gautham Mysore, Luiz Naveda, Uri Nieto, Maria Panteli, Geoffroy Peeters, Colin Raffel, Ranjani H G, Robert Reigle, Martin Rocamora, Gerard Roma, Justin Salamon, Joan Serrà, William Sethares, Umut Şimşekli, T V Sreenivas, Bob Sturm, Julián Urbano, Anja Volk, Frans Weiring and Jose Zapata. My work has been influenced by them in more ways than one, many thanks to all of them and others in the MIR community. I am specially grateful to Dan Ellis, Anssi Klapuri, Aggelos Pikrakis, Matthew Davies and Jason Hockman for providing their code and implementation of the algorithms for evaluation. 

A note of thanks to ERC, DTIC-UPF and Erasmus+ for funding parts of the thesis work. Humble thanks to the maestros of music who have kept the Indian music cultures alive through generations, and the present day music community for being receptive and providing a context for this work. Special thanks to Padma Vibhushan Umayalpuram K. Sivaraman and Pt. Arvind Mulgaonkar for providing their music to build percussion datasets. My sincere thanks go to Vid. G S Nagaraj, to whom I owe all my music training. 

I wholeheartedly thank my flatmates over the years at Mallorca 529: Alastair, Gopal, Sankalp and Shefali (a life coach whose words are collectables), all of whom can be nominated to the best flatmate awards. Special thanks to Eva, whose sense of care and concern goes beyond my comprehension. Thanks to Ratheesh, Geordie, Manoj, Kalpani, Windhya, Waqas, Praveen, Princy, Pallabi and Srinibas for making sunny Barcelona sunnier. 

I am deeply thankful to my parents in India for their unflinching support, daily reminders towards overall well-being, and for bearing with me even on the weeks-long communication blackouts during submission deadlines. Thanks to Sunanda, Veena (the benevolent devil's advocate of my work), Rashmi and friend-philosopher Srinivas for their patient efforts to understand my work. Special thanks to my guide-friend-brother Sharath, who stood 7 hours ahead of me to shield and guide me through matters small and big. And to Soumya, who will be the co-author on all our life's publications in the time to come!\par
\vspace{1em}\par
\noindent Unlike some journeys that end, the research journey never does $\ldots$ \par
\par \par
\noindent Ajay Srinivasamurthy\par
\noindent 29\tsup{th} June 2016
