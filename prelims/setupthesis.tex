% UPF layout, two sides
% A4: (210,297), B5: (176,250)
% If using B5, typically you print in A4 and them use trim.
% So, don't change it.
\ifdefined\PRINTVER
	\setstocksize{297mm}{210mm}  % The size of the stock paper (support), A4
	\settrimmedsize{250mm}{176mm}{*} % B5, the size of the final copy, cut from the stock paper.
	\setlength{\trimtop}{23mm}   % Top margin: top_stock - top_page
	\setlength{\trimedge}{17mm}  % Outer later margin: lat_stock - lat_page
	\setlrmarginsandblock{34mm}{30mm}{*} % 
	\setulmarginsandblock{27mm}{25mm}{*} % 
	\trimLmarks % Ls, not crosses (which may leave traces on the final copy) as trimmarks.
\else
  \setstocksize{250mm}{176mm}  % The size of the stock paper (support), A4
	\settrimmedsize{\stockheight}{\stockwidth}{*} % B5, the size of the final copy, cut from the stock paper.
	\setlength{\trimtop}{0mm}   % Top margin: top_stock - top_page
	\setlength{\trimedge}{0mm}  % Outer later margin: lat_stock - lat_page
	\setlrmarginsandblock{32mm}{32mm}{*} % 
	\setulmarginsandblock{27mm}{25mm}{1.0} % 
	\showtrimsoff
\fi
% Call this to fix layout
\checkandfixthelayout

\maxsecnumdepth{subsection}
\setsecnumdepth{subsection}
\maxtocdepth{subsection}
\settocdepth{subsection}

% Abstract format
\renewcommand{\abstractnamefont}{\large\bfseries}
\renewcommand{\abstracttextfont}{\normalfont}
%
% Some new page styles
% For main matter
\makeatletter
\makepagestyle{mainmatter}
\setlength{\headwidth}{\textwidth}
\makerunningwidth{mainmatter}{\headwidth}
\makeheadrule{mainmatter}{\headwidth}{0pt}
\makeheadposition{mainmatter}{flushright}{flushleft}{}{}
\makepsmarks{mainmatter}{%
  \let\@mkboth\markboth
  \def\chaptermark##1{\markboth{##1}{##1}}    % left mark & right marks
  \def\sectionmark##1{\markright{%
    \ifnum \c@secnumdepth>\z@
      \thesection \ \ %
    \fi
    ##1}}
  \def\tocmark{\markboth{\contentsname}{\contentsname}}%
  \def\lofmark{\markboth{\listfigurename}{\listfigurename}}%
  \def\lotmark{\markboth{\listtablename}{\listtablename}}%
  \def\bibmark{\markboth{\bibname}{\bibname}}%
  \def\indexmark{\markboth{\indexname}{\indexname}}%
}
\makeevenhead{mainmatter}{\normalfont\thepage}{}%\bfseries
                        {\normalfont\scshape\leftmark}
\makeoddhead{mainmatter}{\normalfont\scshape\rightmark}{}%
                       {\normalfont\thepage}%\bfseries
\ifdraftdoc
\makeevenfoot{mainmatter}{\thepage}{}{\textit{Draft: \today, \printtime*{}}}
\makeoddfoot{mainmatter}{\textit{Draft: \today, \printtime*{}}}{}{\thepage}
\fi
\makeatother
% For front matter
\makepagestyle{frontmatter}
\makeevenhead{frontmatter}{\normalfont\thepage}{}{}%\bfseries
\makeoddhead{frontmatter}{}{}{\normalfont\thepage}%\bfseries

\chapterstyle{madsen}
\renewcommand\partnamefont{\normalfont\Huge\sffamily\scshape\raggedleft}
\renewcommand\partnumfont{\normalfont\Huge\sffamily\scshape\raggedleft}
\renewcommand\parttitlefont{\normalfont\HUGE\bfseries\sffamily\raggedleft}
\setsecheadstyle{\Large\bfseries\sffamily}
\setsubsecheadstyle{\large\bfseries\sffamily}
% To add the black chapter box
\addtolength{\marginparwidth}{3mm}   % This is a hack because memoir adds some extra leeway margins
\renewcommand*{\printchapternum}{%
    \makebox[0pt][l]{\hspace{0.4em}%
      \resizebox{!}{4ex}{%
        \chapnamefont\bfseries\sffamily\thechapter}%
    }%
		\marginpar[]{\raggedleft \colorrule[black]{5.8ex}{5.8ex}}
  }%	
\noprelistbreak
% For backmatter
\renewcommand\cftappendixname{\appendixname~}
% A blank page with notes on the top
\newcommand{\blanknotepage}{
\newpage
\thispagestyle{empty}
\begin{center}
-- Notes --
\end{center}
\mbox{}
\newpage
}
% Bibliography
\makeatletter
\newcommand\bibStartNote[1]{\def\@addnote{#1}}
\providecommand\@addnote{}
\patchcmd{\thebibliography}
  {\list}
  {\@addnote\par\list}
  {}
  {}
\makeatother
\bibStartNote{The numbers in brackets at the end of each bibliographic entry indicate the pages in which it is cited.}
% Setting up figures and floats
\renewcommand{\textfraction}{0.01}
\renewcommand{\topfraction}{0.99} 
\renewcommand{\bottomfraction}{0.99}
\renewcommand{\floatpagefraction}{0.95} % floatpagefraction must always be less than the topfraction.
\renewcommand{\abovecaptionskip}{5pt}
% A simple commenting system with colored and highlighted text
\usepackage{soul}
%\newcommand\crule[3][black]{\textcolor{#1}{\rule{#2}{#3}}}
\sethlcolor{yellow}
\def\update#1{{\color{red} #1}}
\ifdefined\DRAFTMODE
	\def\AH#1{{\textbf{\color{green} AH: #1}}}
	\def\AS#1{{\textbf{\color{orange} AS: #1}}}
	\def\comment#1{{\textbf{\color{violet} #1}}}
	\def\note#1{{\textbf{\color{red} #1}}}
	\def\dtext#1{{\textbf{\color{gray} #1}}}
	\def\hlt#1{{\hl{\textbf{#1}}}}
\else
	\def\AH#1{}
	\def\AS#1{}
	\def\comment#1{}
	\def\dtext#1{}
	\def\note#1{}
	\def\hlt#1{}
\fi
% Unnumbered footnotes
\let\svthefootnote\thefootnote
\newcommand\blankfootnote[1]{%
  \let\thefootnote\relax\footnotetext{#1}%
  \let\thefootnote\svthefootnote%
}
% CHANGE-UPF: For using A4
%\settrimmedsize{\stockheight}{\stockwidth}{*} 
%\setlrmarginsandblock{*}{30mm}{1.3} % 
%\setulmarginsandblock{30mm}{*}{1.0} % 
%\captionnamefont{\small\bfseries}
%\captiontitlefont{\normalsize}
%\newlength{\leftspace}
%\setlength{\leftspace}{4cm}
% Some other options for customization. Left for documentation
% \newlength{\umargin}
% \setlength{\umargin}{30mm}
% % \addtolength{\umargin}{\headheight}
% % \addtolength{\umargin}{\headsep}
% \newlength{\lmargin}
% \setlength{\lmargin}{30mm}
% % \addtolength{\lmargin}{\footskip}
%
% \setlength{\trimedge}{\stockwidth}
% \addtolength{\trimedge}{-\paperwidth}
% \addtolength{\trimedge}{-48pt} % 48pt
% \settypeblocksize{48pc}{32pc}{*} % width: Bringhurst 26-29, kth 33
%\setlrmarginsandblock{*}{30mm}{0.75} % change to 1.0 when oneside option
%\setulmarginsandblock{30mm}{*}{1.1} % 
%
%\setheaderspaces{25mm}{*}{*}
% \setmarginnotes{17pt}{51pt}{\onelineskip}
%\setheadfoot{1.5\onelineskip}{2\onelineskip}
%
%\traditionalparskip % For making it zero 
%\abnormalparskip{11pt} % for fixing it to something, uggly
%\nonzeroparskip % probably better than the previous one
%\setlength{\parindent}{0in}