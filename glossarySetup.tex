\usepackage{acronym}
\usepackage[section,numberedsection=autolabel,nomain,acronym,nonumberlist,nogroupskip=true]{glossaries}
%\renewcommand*{\glspostdescription}{}
%\renewcommand*{\bflabel}[1]{{\textbf{\textsf{#1}:}\hfill}}
%\renewcommand*{\glossarysubentryfield}[6]{\glossaryentryfield{##2}{##3}{##4}{##5}{##6}}
\newglossary[lgc]{CM}{gic}{goc}{Carnatic music}  
\newglossary[lgh]{HM}{gih}{goh}{Hindustani music} 
\newglossary[lgt]{TM}{git}{got}{Turkish Makam Music} 
\newglossary[lgb]{BO}{gib}{gob}{Beijing opera (Jingju)}
\newglossary[lgx]{TechTerms}{gix}{gox}{Technical Terms and Datasets} % Unused...
\newignoredglossary{NoShow}
\input{../Papers/commonPool/BOGlossary.tex}
\input{../Papers/commonPool/CarnaticGlossary.tex}
\input{../Papers/commonPool/HindustaniGlossary.tex}
\input{../Papers/commonPool/TurkishGlossary.tex}
\input{../Papers/commonPool/TechGlossary.tex}
%
% Put glsmoveentry to NoShow if you dont want it listed in Glossaries
% \glsmoveentry{tala}{NoShow}
% Set style and make glossaries
\setglossarystyle{index}
% \makeglossaries
% Defining a new style
%%%%%%%%%%%%%%%%%%%%%%%%%%%%%%%%%%%%%%%%%%%%%%%%%%%%%%%%%%%%%%%%%%%%
% New glossary style if needed
\newglossarystyle{myIndex}{%
\setglossarystyle{index}% base this style on the list style
\renewcommand{\glsgroupskip}{}% make nothing happen between groups
\renewcommand*{\glspostdescription}{}
\renewcommand*{\glossentry}[2]{%
\item % bullet point
\glstarget{##1}{\textbf{\glossentryname{##1}}}% the entry name
%\space (\glossentrysymbol{##1})% the symbol in brackets
\space\space\glossentrydesc{##1}% the description
% \space [##2]% the number list in square brackets
}
\renewcommand*{\subglossentry}[3]{%
\item % bullet point
\hspace{4ex} \glstarget{##2}{\textbf{\glossentryname{##2}}}% the entry name
%\space (\glossentrysymbol{##1})% the symbol in brackets
\space\space\glossentrydesc{##2}% the description
% \space [##2]% the number list in square brackets
}
}% End of myIndex
%\makeindex
%\glossarystyle{index}
\setglossarystyle{myIndex}
\renewcommand{\glsglossarymark}[1]{%
\markright{#1}
% \ifglsucmark
% \markright{\MakeTextUppercase{#1}}%
% \else
% \markright{#1}%
% \fi
}
\makeglossaries